\documentclass[a4paper,10pt]{article}

\usepackage{hyperref}

%opening
\title{Supplementary methods: tutorial for the modularization of the G1/S network using BiNoM functions}
\author{}

\begin{document}

\maketitle

\begin{abstract}
In this tutorial, we describe the step-by-step procedure to extract a compact modular network from a large network representing the cell cycle G1/S transition. The input file can be downloaded from our website \url{http://bioinfo-out.curie.fr/projects/binom/}.
\end{abstract}

%\section{}

\begin{enumerate}
\item Import network. \\
\textbf{Plugins $\rightarrow$ BiNoM I/O $\rightarrow$ Import CellDesigner document
from file}.  \\
Choose G1S.xml

% fournir le fichier xml correspondant

\item	Decompose the network into subnetworks. \\
We choose to decompose the network according to the different components present
in the map rather than decomposing according to the cycles of the network. 
\textbf{Plugins $\rightarrow$ BiNoM Analysis $\rightarrow$ Get Material
Components…}.\\
36 subnetworks are created.

\item	Clustering. \\
The resulting subnetworks may share a lot of species. To reduce the number of
subnetworks, we cluster them with an overlap of 25\%. \\
\textbf{Plugins $\rightarrow$ BiNoM Analysis $\rightarrow$ Cluster networks}. \\
A pop-up window appears. Among the proposed networks, select all networks except
G1S.xml and choose 25\% overlap. 7 networks are created.

\item	Rename the newly-created clusters.
It is important to rename right away the clusters with a name than illustrates
the content of each of these clusters. 
Choosing the appropriate name might be facilitated by listing the name of the
components of the cluster: \textbf{Plugins $\rightarrow$ BiNoM module manager
$\rightarrow$ List components of species in network and modules} \\
To rename them, right-click on the Network panel.
We propose to use the following names for these 7 clusters: \textbf{E2F1\_RB},
\textbf{CycD1}, \textbf{CycE\_A}, \textbf{Wee1}, \textbf{p21}, \textbf{CycH} and
\textbf{CycB1}.


\item	Inside the clusters, check the content of the modules (manual curation
of the modularization).

\begin{itemize}
\item	For cluster E2F1\_RB, separate RB from E2F1 and from E2F6 and create
three different modules. \textbf{Select - Nodes - By Name}. Write E2F6. The
nodes with E2F6 are highlighted.Select the reactions between the nodes.
\textbf{Select - Nodes - First neighbours of selected nodes}. \textbf{File - New
- Network - From selected nodes all edges}. Rename E2F1\_RB child Delete the
nodes from E2F1\_RB network.
\item	Do the same for pRB.
\item	Rename the remaining network of E2F1\_RB, E2F1.
\item	For cluster, CycB1, select CDC25 and create a new module with components
including CDC25.
\item	The resulting list of modules is:CDC25, CycB1, Wee1, CycH, p21, CycD1,
CycE\_A, E2F1, RB and E2F6.
\end{itemize}

\item	Select modules to prepare the modular view. In the pop-up window, choose
the 10 modules. \textbf{Plugins - BiNoM modules - Create nest network}.
\item	Connect modules. Plugins - BiNoM modules - Update connections between
nests.
\item   Verify the resulting modular view

\begin{itemize}
\item	Merge networks (modules). \textbf{Plugins - Advanced network merge}.
Choose union and the 10 modules. Resulting network: 209 nodes and 476
edgesInitial network: 209 nodes and 266 edges LC: there seems that there was a
duplication of edges in the merging process.
\item	Update connections from other network.
\item	Network difference.
\end{itemize}

\end{enumerate}

\end{document}
