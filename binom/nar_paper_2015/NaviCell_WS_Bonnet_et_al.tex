\documentclass[a4,center,fleqn]{NAR}

% Enter dates of publication
\copyrightyear{2008}
\pubdate{31 July 2009}
\pubyear{2009}
\jvolume{37}
\jissue{12}

%\articlesubtype{This is the article type (optional)}


\begin{document}

\title{NaviCell Web service for Network-based Data Visualization}


\author{%
Eric Bonnet\,$^{1,2,3}$,
Eric Viara\,$^{4}$,
Inna Kuperstein\,$^{1,2,3}$,
Laurence Calzone\,$^{1,2,3}$,
David PA Cohen\,$^{1,2,3}$,
Emmanuel Barillot\,$^{1,2,3}$,
Andrei Zinovyev\,$^{1,2,3}$%
\footnote{To whom correspondence should be addressed.
Tel: +33 (0)1 56 24 69 89; Fax: +33 (0)1 56 24 69 11; Email: andrei.zinovyev@curie.fr}}


%\author{%
%Corresponding Author\,$^{1,*}$,
%First Co-Author\,$^{2}$
%and Second Co-Author\,$^2$%
%\footnote{To whom correspondence should be addressed.
%Tel: +44 000 0000000; Fax: +44 000 0000000; Email: xxx@yyyy.ac.zz}}

\address{%
$^{1}$Institut Curie, 26 rue d'Ulm, 75248 Paris, France, 
$^{2}$INSERM U900, 75248 Paris, France,
$^{3}$Mines ParisTech, 77300 Fontainebleau, France,
$^{4}$Sysra, 91330 Yerres, France.
}
% Affiliation must include:
% Department name, institution name, full road and district address,
% state, Zip or postal code, country

\history{%
Received January 1, 2009;
Revised February 1, 2009;
Accepted March 1, 2009}


\bibliographystyle{nar}
\maketitle

\begin{abstract}
Text. Text. Text. Text. Text. Text. Text. Text. Text. Text. Text.
\end{abstract}


\section{Introduction}

Text. Text. Text. Text. Text. Text. Text. Text. Text. Text. Text.

%\begin{align*}
%&\mathrm{Ascorbate} + \mathrm{EDTA} \cdot \mathrm{Fe}^{3+} \to
%\hbox{Oxidized ascorbate}
%\\
%&\mathrm{EDTA} \cdot \mathrm{Fe}^{2+} + \mathrm{H}_2
%\mathrm{O}_2 \to
%\mathrm{EDTA} \cdot \mathrm{Fe}^{3+} + \cdot
%\mathrm{OH} + \mathrm{OH}^-
%\end{align*}

Text. Text. Text. Text. Text. Text. Text. Text. Text. Text. Text.
Text. Text. Text. Text. Text. Text. Text. Text. Text. Text. Text.
Text. Text. Text. Text. Text. Text. Text. Text. Text. Text. Text.
Text. Text. Text. Text. Text. Text. Text. Text. Text. Text. Text.

Text. Text. Text. Text. Text. Text. Text. Text. Text. Text. Text.
Text. Text. Text. Text. Text. Text. Text. Text. Text. Text. Text.
Text. Text. Text. Text. Text. Text. Text. Text. Text. Text. Text.
Text. Text. Text. Text. Text. Text. Text. Text. Text. Text. Text.
Text. Text. Text. Text. Text. Text. Text. Text. Text. Text. Text.
Text. Text. Text. Text. Text. Text. Text. Text. Text. Text. Text.
Text. Text. Text. Text. Text. Text. Text. Text. Text. Text. Text.
Text. Text. Text. Text. Text. Text. Text. Text. Text. Text. Text.
Text. Text. Text. Text. Text. Text. Text. Text. Text. Text. Text.
Text. Text. Text. Text. Text. Text. Text. Text. Text. Text. Text.
Text. Text. Text. Text. Text. Text. Text. Text. Text. Text. Text.
Text. Text. Text. Text. Text. Text. Text. Text. Text. Text. Text.

% **************************************************************
% Keep this command to avoid text of first page running into the
% first page footnotes
%\enlargethispage{-65.1pt}
% **************************************************************

Text. Text. Text. Text. Text. Text.
Text. Text. Text. Text. Text. Text. Text. Text. Text. Text. Text.
Text. Text. Text. Text. Text. Text. Text. Text. Text. Text. Text.
Text. Text. Text. Text. Text. Text. Text. Text. Text. Text. Text.
Text. Text. Text. Text. Text. Text. Text. Text. Text. Text. Text.
Text. Text. Text. Text. Text. Text. Text. Text. Text. Text. Text.
Text. Text. Text. Text. Text. Text. Text. Text. Text. Text. Text.
Text. Text. Text. Text. Text. Text. Text. Text. Text. Text. Text.
Text. Text. Text. Text. Text. Text. Text. Text. Text. Text. Text.
Text. Text. Text. Text. Text. Text. Text. Text. Text. Text. Text.
Text. Text. Text. Text. Text. Text. Text. Text. Text. Text. Text.
Text. Text. Text. Text. Text. Text. Text. Text. Text. Text. Text.
Text. Text. Text. Text. Text. Text. Text. Text. Text. Text. Text.
Text. Text. Text. Text. Text. Text. Text. Text. Text. Text. Text.
Text. Text. Text. Text. Text. Text. Text. Text. Text. Text. Text.
Text. Text. Text. Text. Text. Text. Text. Text. Text. Text. Text.
Text. Text. Text. Text.
Text 

\section{ALGORITHMS AND SOFTWARE}

\subsection{Implementation}

The NaviCell software tool is implementing a set of JavaScript functions on a
web interface, allowing an easy and intuitive navigation of large-scale
molecular maps of biological networks \cite{kuperstein2013navicell}. We
conceived the NaviCell Web Service data visualization tool as an extension of
NaviCell allowing users to visualize and analyze different types of "omics"
data. The Web Service is implemented at two different functional levels that are
interconnected. The first level corresponds to the interactive, windows-based,
usage of the web interface to load and visualize data. This level is implemented
as a specific set of JavaScript functions linked to a button-based menu located
in the right-hand panel of the web interface. The second level corresponds to a
programmatic usage of the service, which allow users to write programs that will
communicate with the NaviCell \emph{server} in order to automate all the
visualization operations. This level is implemented as a REST 


\subsubsection{Materials subsubsection one.}

Text. Text. Text. Text. Text. Text. Text. Text. Text. Text. Text.
Text. Text. Text. Text. Text. Text. Text. Text. Text. Text. Text.
Text. Text. Text. Text:
%\begin{align}
%\mathrm{LD}^r = \frac{\mathrm{LD}}{A_\mathrm{iso}}
%= 1.5 S \left( 3 \cos^2 \alpha_i - 1 \right)
%\end{align}
Text. Text. Text. Text. Text. Text. Text. Text. Text. Text. Text.
Text. Text. Text. Text. Text. Text. Text. Text. Text. Text. Text.
Text. Text. Text. Text. Text. Text. Text. Text. Text. Text. Text.
Text. Text. Text. Text. Text. Text. Text. Text. Text. Text. Text.
Text. Text. Text. Text. Text. Text. Text. Text. Text. Text. Text.
Text. Text. Text. Text. Text. Text. Text. Text. Text. Text. Text.
Text. Text. Text. Text. Text. Text. Text. Text. Text. Text. Text.
Text. Text. Text. Text. Text. Text. Text. Text. Text. Text. Text.
Text. Text. Text. Text. Text. Text. Text. Text. Text. Text. Text.
Text. Text. Text. Text. Text. Text. Text. Text. Text. Text. Text.
Text. Text. Text. Text. Text. Text. Text. Text. Text. Text. Text.
Text. Text. Text. Text. Text. Text. Text. Text. Text. Text. Text.


\section{RESULTS}

\subsection{Results subsection one}

Text. Text. Text. Text. Text. Text. Text. Text. Text. Text. Text.
Text. Text. Text. Text. Text. Text. Text. Text. Text. Text. Text.
Text. Text. Text. Text. Text. Text. Text. Text. Text. Text. Text.
Text. Text. Text. Text. Text. Text. Text. Text. Text. Text. Text.
Text. Text. Text. Text. Text. Text. Text. Text. Text. Text. Text.
Text. Text. Text. Text. Text. Text. Text. Text. Text. Text. Text.
Text. Text. Text. Text. Text. Text. Text. Text. Text. Text. Text.
Text. Text. Text. Text. Text. Text. Text. Text. Text. Text. Text.
Text. Text. Text. Text. Text. Text. Text. Text. Text. Text. Text.
Text. Text. Text. Text. Text. Text. Text. Text. Text. Text. Text.
Text. Text. Text. Text. Text. Text. Text. Text. Text. Text. Text.
Text. Text. Text. Text. Text. Text. Text. Text. Text.

\section{DISCUSSION}

\subsection{Discussion subsection one}

Text. Text. Text. Text. Text. Text. Text. Text. Text. Text. Text.

\subsection{Discussion subsection two}

Text. Text. Text. Text. Text. Text. Text. Text. Text. Text. Text.

\section{CONCLUSION}

Text. Text. Text. Text. Text. Text. Text. Text. Text. Text. Text.

\section{ACKNOWLEDGEMENTS}

Text. Text. Text. Text. Text. Text. Text. Text. Text. Text. Text.


\subsubsection{Conflict of interest statement.} None declared.
\newpage

\bibliography{biblio}

%\begin{thebibliography}{4}
%
%% Format for article
%\bibitem{1}
%Author,A.B. and Author,C. (1992)
%Article title.
%\textit{Abbreviated Journal Name}, \textbf{5}, 300--330.
%
%% Format for book
%\bibitem{2}
%Author,D., Author,E.F. and Author,G. (1995)
%\textit{Book Title}.
%Publisher Name, Publisher Address.
%
%% Format for chapter in book
%\bibitem{3}
%Author,H. and Author,I. (2005)
%Chapter title.
%In
%Editor,A. and Editor,B. (eds),
%\textit{Book Title},
%Publisher Name, Publisher Address,
%pp.\ 60--80.
%
%% Another article
%\bibitem{4}
%Author,Y. and Author,Z. (2002)
%Article title.
%\textit{Abbreviated Journal Name}, \textbf{53}, 500--520.
%
%\end{thebibliography}
%

\end{document}
