\documentclass[a4,center,fleqn]{NAR}
%\usepackage{hyperref}

\usepackage[pdftex,
colorlinks=true,
citecolor=black,
urlcolor=blue
]{hyperref}


% Enter dates of publication
\copyrightyear{2008}
\pubdate{31 July 2009}
\pubyear{2009}
\jvolume{37}
\jissue{12}

%\articlesubtype{This is the article type (optional)}


\begin{document}

\title{NaviCell Web Service for Network-based Data Visualization}


\author{%
Eric Bonnet\,$^{1,2,3}$,
Eric Viara\,$^{4}$,
Inna Kuperstein\,$^{1,2,3}$,
Laurence Calzone\,$^{1,2,3}$,
David PA Cohen\,$^{1,2,3}$,
Emmanuel Barillot\,$^{1,2,3}$,
Andrei Zinovyev\,$^{1,2,3}$%
\footnote{To whom correspondence should be addressed.
Tel: +33 (0)1 56 24 69 89; Fax: +33 (0)1 56 24 69 11; Email: andrei.zinovyev@curie.fr}}

\address{%
$^{1}$Institut Curie, 26 rue d'Ulm, 75248 Paris, France,
$^{2}$INSERM U900, 75248 Paris, France,
$^{3}$Mines ParisTech, 77300 Fontainebleau, France,
$^{4}$Sysra, 91330 Yerres, France.
}
% Affiliation must include:
% Department name, institution name, full road and district address,
% state, Zip or postal code, country

\history{%
Received January 1, 2009;
Revised February 1, 2009;
Accepted March 1, 2009}


\bibliographystyle{nar}
\maketitle

\begin{abstract}
NaviCell Web Service is a tool for network-based visualization of
``omics" data.  It implements several data visual representation methods,
including the novel map staining technique for grasping large-scale
trends in numerical values (such as whole transcriptome) projected on top of a
pathway map. NaviCell Web Service allows combining visualization of different data types.
It is scalable, based on Google Maps technology allowing working with large pathway maps,
containing thousands of nodes, using semantic zooming principles.
NaviCell Web Service can be applied to pathway maps of different
types represented in various formats in different application fields.
The web service provides a server mode, which allows automating
visualization tasks and retrieve data from maps via RESTfull (standard HTTP)
calls.  Bindings to different programming languages are provided (Python,
R, Java). We illustrate the purpose of the tool  with  several case studies using pathway maps created by different research groups,
in which data visualization serves for getting new insights into molecular mechanisms
involved in systemic disease progression (cancer and neurodegenerative disease).
\end{abstract}


\section{Introduction}

Biology is a scientific discipline deeply grounded in visual representations
serving for communicating results and ideas. Nowadays, there is a strong
incentive to provide \emph{interactive} web-based visual representations, where
users can easily adapt them to address a
particular biological question. Modern molecular biology is particularly
demanding for new tools that can represent numerous ``omics" data in a
meaningful way.  Visualizing quantitative ``omics" data in the context of
biological networks provides insights into the molecular mechanisms in healthy
tissues and in diseases \cite{Gehlenborg2010,Barillot2012}. This is one of the
most demanded features of existing pathway databases such as KEGG and Reactome
\cite{Kanehisa2012, Croft2014}. To answer this need, many tools have been
developed allowing mapping ``omics" data on top of biological networks
\cite{Arakawa2005,vanIersel2008,Luo2013, Nishida2014}. These tools use the
content of existing pathway databases to retrieve the pathway information
through dedicated APIs. In addition, some pathway databases provide graphical
user interfaces (GUI) allowing performing data visualization, on interactive pathway maps.

When a user faces the necessity to visualize the ``omics" data on
biological networks, there are currently only two options available: either to use
a graphical interface of a suitable pathway database, which can be a tedious manual work, or use
one of the available standalone or web-based tools for creating static images
of colored pathways which are not interactive anymore, and do not allow browsing
the molecular interactions. Currently, there is a lack of well-developed APIs
that allow to apply data visualization programmatically on biological
networks, such that the ``omics" data can be browsed simultaneously with the
pathway information.

Available network-based methods for data visualization have several common
limitations. First, most of them do not provide any possibility of data
abstraction and do not allow visualizing coarse-grained trends in the ``omics" data.
This is needed if, for example, there is a wish to visualize the whole
transcriptome of a cell or a group of samples on a large network,
representing a large part of the cellular interactome. Current methods usually attach
some elements of standard scientific graphics (gradient color, heatmaps,
barplots) to the individual elements of pathway maps, which makes the
visualization hardly readable at higher levels of zooming as well as the map
content itself. Second, most of data visualization tools are specific to a
particular pathway database structure, format and graphical pathway
representation. This limits the use of data visualization on user-defined maps.
Third, APIs for programmatic web-based data
visualization are very limited or do not exist.

To overcome some of these limitations, we have developed NaviCell Web Service tool. It allows
visualizing ``omics" data via GUI and flexible API using standard and advanced
methods of data visualization, including a possibility to perceive the mapping of data
at different network scales. The NaviCell Web Service allows
exploiting pathway maps created by users or from existing databases,
including large network maps.


% **************************************************************
% Keep this command to avoid text of first page running into the
% first page footnotes
\enlargethispage{-65.1pt}
% **************************************************************

The NaviCell Web Service for Network-based Data
Visualization is a combination of 1) a user-friendly NaviCell JavaScript-based
web interface \cite{kuperstein2013navicell}, allowing browsing large maps
of biological networks using Google Maps API and \emph{semantic zooming}
principle and visualize ``omics" data on top of them
(https://navicell.curie.fr); 2) a collection of maps available for
data visualization including the Atlas of Cancer Signaling Network (ACSN)
(https://acsn.curie.fr)  and other detailed network maps created by other
research groups for particular topics (Toll-like receptor signalling, EGFR
and mTOR pathways, mast cell activation, dendritic cells, iron metabolism,
Alzheimer disease and others); 3) a REST API allowing programmatic use of all
data visualization functions and manipulating the web interface, with available
Python binding and upcoming R and Java bindings.

A quick ``live example'' which \emph{automatically} uploads a sample dataset from
TCGA database and illustrates the data visualization capabilities of
NaviCell Web Service is provided at
http://navicell.curie.fr/pages/nav\_web\_service.html. Sample Python
code for quick start on using NaviCell Web Service API is available from the
same link in the ``Python API and data files" section. An example screenshot of NaviCell
Web Service GUI can be found in the manual located at the same link.

\section{ALGORITHMS AND SOFTWARE}

\subsection{Implementation}

We conceived the NaviCell Web Service for Network-based Data Visualization
as a significant development of NaviCell \cite{kuperstein2013navicell}
for visualizing and analyzing different types of ``omics" data. The web service
is implemented at two interconnected functional levels.
The first level corresponds to the interactive, dialog-based,
use of the web interface to load and visualize data. This level is implemented
as a specific set of JavaScript functions linked to the menu located
in the right-hand panel of the NaviCell web interface. The second level corresponds to a
programmatic usage of the service, which allows users writing code that will
communicate with the NaviCell \emph{server} in order to automate all the
visualization operations. This functional level is implemented as a RESTful web
service, using standard web protocol HTTP operations and data encoding (JSON)
to perform all the necessary operations \cite{fielding2002principled}.
RESTful web services are very popular, have the advantage of being
lightweight, efficient and simple to use and allows implementation in different programming languages. 
Complete API for NaviCell Web Service is implemented in Python, and
partial versions of APIs for R and Java are available. These APIs will facilitate the integration
of NaviCell Web Service in other applications: among them, there exists an
ongoing work on providing the data visualization services through the Garuda
Alliance web platform, which aims to be a one-stop service for bioinformatics
and systems biology \cite{ghosh2011software} and GeneSpring software
\cite{Chu2001}. Figure 1 is summarizing the NaviCell \emph{server} architecture
and information flow.

\subsection{Input data types}

The pathway maps which can be used for NaviCell Web Service data visualization
can be prepared using CellDesigner tool \cite{Funahashi2008}, which is widely used in systems biology
community and implements the Systems Biology Graphical Notation standard \cite{LeNovere2009}
for representing biological networks. These pathway maps are converted into
interactive Google Maps-based web-interfaces via NaviCell tool \cite{kuperstein2013navicell}.
Other pathway map formats can be used in NaviCell through dedicated converters implemented
in BiNoM Cytoscape plugin \cite{Zinovyev2008,Bonnet2013a}.
This method allows using with NaviCell practically any network that can be imported into Cytoscape environment.

NaviCell data visualization web service is able to process
several types of ``omics" data. The complete list of data types that can be used in NaviCell Web Service
is provided in Table \ref{table:01}. The different biological data
types are mapped into several internal representations that determine what methods
of data visualization can be applied.

For instance, an mRNA expression data matrix is associated to a
``continuous" numerical internal representation. Thus, if the user is choosing to display
this data with a heatmap, mapping to a color gradient will be applied by default, with a possibility
to modify the default settings. On the other hand, when a
matrix with discrete copy-number data is loaded, it is associated with a
``discrete ordered" internal representation for which
a specific color palette is applied for visualization,
with a distinct color associated to each discrete copy-number state.

The input format for data sets is standard tab-delimited text files, with rows
representing genes or their products and columns representing samples (or experiments, or time
points). Genes in the first column should be labeled by their standard HUGO (HGNC)
gene symbols, that will be associated to the different entities (genes,
proteins, complexes) on the pathway map.

Users can also upload sample annotation files to specify how samples can be organized into meaningful groups
(e.g., disease vs control), as a simple tab-delimited text file with sample names in
rows and annotation fields as columns. An appropriate method will be used
to summarize the values of all the samples contained in a group, defined by the
internal data representation. For instance, expression values for a group of samples
is averaged by default for the ``continuous" data type, but the user will have
a possibility to change the default method and use the median or
the maximum value in the group, etc. For mutation data (``discrete unordered" internal data type),
taking the average does not make sense and instead the grouping method by default is
``at least one element of the group is mutated".

%\begin{table}[b]
\begin{table}
\tableparts{%
\caption{List of biological input data types for NaviCell Web Service data visualization. The
first column lists the types as they appear in the NaviCell Web Service interface. The
second column lists the internal data representations that are used to determine what
type of data visualization can be applied for a selected data type.}
\label{table:01}%
}{%
\begin{tabular*}{\columnwidth}{@{}ll@{}}
\toprule
``Omics" data type &  Internal representation
%\\
%& (\%) & (s$^{-1}$) & (\%) & (s$^{-1}$)
\\
\colrule
mRNA expression data & continuous
\\
microRNA expression data & continuous
\\
Protein expression data & continuous
\\
Discrete copy number data & discrete ordered
\\
Continuous copy number data & continuous
\\
Mutation data & discrete unordered
\\
Gene list & set
\\
\botrule
\end{tabular*}%
}
{
}
\end{table}

\subsection{Graphical data representations}

We have included in the NaviCell Web Service for network-based data visualization several methods
for graphical representation of molecular data. Some of them are standard
and broadly used in molecular biology (heatmaps and barplots), while others (such as map
staining) have not been previously employed in the context of network-based
data visualization.

\begin{itemize}

\item \textbf{Simple markers} are pictograms (similar to the ones used in Google Maps
to indicate the geographical locations) drawn on the pathway map at the location
of a given molecular species (protein, gene, complex, phenotype). They are used to display the
results of a search performed by the user or to display a list of gene names 
mapped onto the biological network 
(including species that are part of a complex, or different forms of a given protein).

\item \textbf{Heatmaps} display individual values of a data matrix as
colors. They are often used in molecular biology, in particular for displaying expression
data. In NaviCell Web Service, heatmaps can be used to
visualize continuous data types such as expression values, or discrete data such
as copy-number or mutation values. The user can arrange the visualization to show
several samples or groups of samples from several data sets
(for example, for showing simultaneously gene promoter methylation and expression values).

\item \textbf{Barplots} are charts with rectangular bars proportional to the values
they represent. On NaviCell Web Sevice interface, they are colored according to the
data value. Barplots can be efficient to visually distinguish numerical
values between two or more groups of conditions (e.g. disease vs control as in Figure 3B-D).

\item \textbf{Glyphs} are graphical representations using basic geometrical
shapes (triangle, square, rectangle, diamond, hexagon, circle) and their attributes
to visually combine different types of data. The user can specify
what data to map on the shape, the size and the color of the glyph.
For example, one might consider using matched data for
gene copy number and expression. In this case, the shape of the
glyphs can be assigned to the copy-number values, while the expression data
would be used for the color of the glyph. In this manner, users can quickly
grasp the two data types values mapped on the same glyph.
Up to five glyphs can be attached to the same molecular entity,
which can be useful to compare several experimental conditions on the same
map.

\item \textbf{Map staining} is a novel network-based data visualization method where
background areas (or territories) around each molecular entity are colored according to the data
value associated to this entity. The area occupied by a particular map element is defined as the Voronoi's cell
associated to this element. The Voronoi's cell is a convex polygon containing all the points
of the territory which are closer to a chosen element than to any other element
\cite{aurenhammer1991voronoi}. In our case, we use all the entities present on a pathway map
for defining the Voronoi's cells. The polygons are pre-computed for each map and post-processed
to avoid too large polygons, and are colored according to the data values and to the options
defined by the user. The colorful decorations of the pathway map
are replaced by a black-and-white background in order to avoid mixing the colors.
Using map staining allows appreciating large-scale trends in molecular data mapped
onto biological networks, when observed at the bird's-eye view level (see example in Figure 2 and 3).

\end{itemize}



\section{RESULTS}

We show that NaviCell Web Service can be used for simultaneous analysis of
different types of high-throughput data in two case studies:
\begin{enumerate}
\item comparing two prostate cancer cell lines using the cell cycle map,
focusing on transcriptomics and mutation data.

\item using the map of molecular interactions involved in Alzheimer disease
\cite{Mizuno2012} for visualizing the transcriptome data collected for
different brain areas \cite{hokama2013altered}.
\end{enumerate}

Furthermore, several examples of using NaviCell Web Service for data
visualization can be found in NaviCell Web Service user guide and case studies
provided at http://navicell.curie.fr/pages/nav\_web\_service.html. In Supplementary
Materials we provide two case studies demonstrating visualization of ovary cancer data obtained from The Cancer Genome
Atlas \cite{TCGA2011Ovarian} on a large map of Atlas of Cancer Signalling
Network \cite{Kuperstein2015} and an example of using non-CellDesigner network map
of the Ewing's sarcoma signalling network \cite{Stoll2013} for visualizing transcriptomic time series data.


\subsection{Comparing two prostate cancer cell lines}

As an illustration of the use of maps in comparing two cancer genomic profiles, we
selected two prostate cancer cell lines from the Cancer Cell Line Encyclopedia
\cite{barretina2012cancer}: a prostate hormone-sensitive
tumor cell line (LNCaP), and a prostate hormone-refractory tumor cell line
(DU145). We gathered gene expression, copy number and mutation data for
these two cell lines and mapped them onto the cell cycle map
\cite{calzone2008comprehensive}.

Application of map staining data visualization technique allows concluding that
most LNCaP cells express genes from the early phase G1 (Figure 2A, upper-right area) and the G1-S checkpoint
(Figure 2A, lower-left area) while most DU145 cells express genes from the
later stages of the cell cycle (Figure 2B, G1-late, S-phase and G2-phase
areas). Gene set visualization shows that few cell cycle genes are mutated, amplified or lost
in these two cell lines (Figure 2A and 2B, square and triangle glyphs, especially in DU145).
By zooming in on the map, we can determine that for the LNCaP cells, the most
important gene alterations are the amplification of DP1 (present in most
complexes involving E2F1), and the homozygous loss of E2F2 transcription
factor. The mutations concern mainly genes from the apoptotic pathway: ATR or
CHEK2. As for the expression, the more noticeable trend is that the expression
of the cell cycle inhibitors, such as RBL2, p21, CDC25C, is still high, whereas
the expression of the G2 cyclin, for instance, CyclinA is low.

For the DU145 cells, genes involved in later stages of the cell cycle are more
expressed compared to the LNCaP cells. It can be that if the cells were
arrested in LNCaP, they are more advanced in the cycle in DU145 cells by
overpassing the G1/S checkpoint and are arrested at the spindle checkpoint.
They are more prone to proliferation than the LNCaP cells. The expression of some
cyclins seems to confirm this fact: CyclinB, CyclinD and CyclinH are higher
than in LNCaP. Some means to stop the cycle seem to be kept though, with Cdc20
and Cdc25 expression high.

%We checked the expression levels of KI-67, a marker of proliferation, in the
%dataset and its expression is indeed higher in DU145 than in LNCaP cells, which
%tends to confirm the hypothesis that DU145 are more proliferative than LNCaP
%cells.

\subsection{Visualization of transcriptomes of brain cortex regions in Alzheimer's disease}

Here we used Alzheimer's disease network map (AD \cite{Mizuno2012}) to visualize
the expression data from AD study \cite{hokama2013altered} (Figure~3) and mutation data on most frequently
mutated genes in AD (taken from http://www.alzforum.org,
http://www.molgen.ua.ac.be). The AD/control expression ratio from
Thalamocortical area (TC) is displayed as map staining, and glyphs identify
mutated genes. The data is visualized at the top level view of the map, in
order to grasp patterns of expression regulation and distribution of mutated
genes across different molecular mechanisms depicted on the map as functional
modules (Figure~3A). At the detailed zoom level, the bar plot display mode is
used to compare expression values in
samples from Frontal cortex (FC), Thalamocortical area (TC) and Hippocampus
(HI) areas (Figures~3B and 3D).  Zoom in on the most deregulated mechanisms in
AD shows elevated apoptosis signaling as overexpression of caspase cascade
(Figure~3B) and downregulated cell survival mechanisms as MAPK pathway (Figure~3D),
which are known facts. In addition, AD
samples are enriched with vascular inflammatory genes as ICAM1, VCAM1, EDN1 and
TNF, the evidences of Blood-brain barrier (BBB) breakdown and inflammatory
immune response activation in the brain tissue (Figure~3C).

\section{DISCUSSION}

We have compared the features of the NaviCell Web Service for data visualization
with similar tools. We have selected a number of web-based tools that are providing
similar functionalities, i.e. easy pathway browsing functions and interactive data
visualization capabilities, and compared them with NaviCell Web service. We have focused on the features
related to map navigation, molecular data types, graphical representations,
and programmatic access. The results of this comparison show that the NaviCell Web Service's strength is in
increased possibilities for graphical representations, support of multiple
data types and extended possibilities for programmatic access from different
computer languages (Table~2).
%To extend the scope of NaviCell Web Service, we are
%currently working on porting some of the most used pathway databases into the
%CellDesigner format which can be used after for display in NaviCell, with a
%possibility of data visualization.


%\begin{table}[b]
\begin{table}
\tableparts{%
\caption{Comparison of the NaviCell Web Service features with similar web sites for pathway-based data visualization.
}
\label{table:02}%
}{%
\begin{tabular*}{\columnwidth}{@{}lcccccc@{}}
\toprule
 Features & Na & Re & KE & iP & Bc & Pa
%\\
%& (\%) & (s$^{-1}$) & (\%) & (s$^{-1}$)
\\
\colrule
Map: navigation & $\bullet$ & $\bullet$ & & $\bullet$ & $\bullet$ & $\bullet$
\\
Map: simple zooming & $\bullet$ & $\bullet$ & $\bullet$ & $\bullet$ & $\bullet$ &  $\bullet$
\\
Map: semantic zooming & $\bullet$ & & & & &
\\
Visualization: node coloring & & $\bullet$ & & & & $\bullet$
\\
Visualization: heatmaps & $\bullet$ & & & & $\bullet$ &
\\
Visualization: barplots & $\bullet$ & & & & $\bullet$ &
\\
Visualization: glyphs & $\bullet$ & & & & &
\\
Visualization: map staining & $\bullet$ & & & & &
\\
Data mapping: gene lists& $\bullet$ & $\bullet$ & $\bullet$ & $\bullet$ & $\bullet$ & $\bullet$
\\
Data mapping: expression data& $\bullet$ & $\bullet$ & & & $\bullet$ & $\bullet$
\\
Data mapping: copy-number data& $\bullet$ & & & & &
\\
Data mapping: mutation data& $\bullet$ & & & & &
\\
Data mapping: metabolomic data& & $\bullet$ & & & &
\\
Data mapping: interactions & & $\bullet$ & & & &
\\
Programmatic access: RESTful web & $\bullet$ & $\bullet$ & $\bullet$ & & $\bullet$ &
\\
Programmatic access: data visual. & $\bullet$ & & & & $\bullet$ &
\\
\botrule
\end{tabular*}%
}
{Abbreviations: Na: NaviCell Web Service, Re: Reactome\cite{Croft2014},
KE: KEGG\cite{Kanehisa2012}, iP: iPath\cite{letunic2008ipath}, Bc:
BioCyc\cite{karp2005expansion}, Pa: PATIKAweb\cite{demir2002patika}.
}
\end{table}


\section{CONCLUSION}

NaviCell Web Service shall contribute to the growing set of highly demanded
tools for molecular biology allowing visualization of ``omics" data in the
context of biological network maps.

\section{ACKNOWLEDGEMENTS}

This work is supported by Agilent Thought Leader Award \#3273, by ASSET European Union Framework Program 7 project under grant agreement FP7-HEALTH-2010-259348 and by the
grant ``Projet Incitatif et Collaboratif Computational Systems Biology Approach for Cancer" from Institut Curie.

\subsubsection{Conflict of interest statement.} None declared.

\section{FIGURE LEGENDS}

\textbf{Figure 1.} General architecture of the NaviCell Web service
\emph{server}. Client software (light blue layer) communicates with the server
(red layer) through standard HTTP requests using the standard JSON format to
encode data (RESTful web service, dark blue layer). A session (with a unique ID)
is established between the server and the browser (yellow layer) through
Ajax communication channel to visualize the results of the commands send by the
software client. It is worth noticing that communication channels are bidirectional, i.e.
the client software can send data (e.g. an expression data matrix) to the
server, but it can also receive data from the server (e.g. a list of HUGO
gene symbols contained in a map).

\textbf{Figure 2.} Visualization of multiple data types for two different
prostate cancer cell lines. Transcriptomics, gene copy-number values and gene
mutations \cite{barretina2012cancer} are mapped on the Cell Cycle map. (A)
hormone-sensitive prostate cancer cell line (LNCaP). (B) hormone-refractory
prostate cancer cell line (DU145). Expression data is visualized by using the
map staining technique, i.e. colored territories around entities, ranging from
low  (green) to high expression values (red).  Copy-number values are
represented by glyphs (squares) with blue color indicating gene loss (values of
-1 and lower) and yellow color indicating amplification (values of 1 and
higher). Mutated genes are depicted by cyan triangles.

\textbf{Figure 3.} Visualization of two different data types on Alzheimer's
disease (AD) pathway map.  (A) global map, expression data is visualized with
map staining (see Figure 2 legend), while frequently mutated genes are indicated by
glyphs (blue triangles), with the size of the glyph proportional to the
mutation frequency. (B-C-D) represent zoom on known key regulators of (B)
apoptosis, (C) blood brain barrier and (D) MAPK signaling pathway. Background
color represent expression values as in (A), while barplots illustrate tissue
specific expression values for frontal cortex (FC), thalamo-cortical area (TC)
and hippocampus (HI).

\bibliography{biblio}

\end{document}
