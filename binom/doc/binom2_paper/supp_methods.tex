\documentclass[a4paper,10pt]{article}

\usepackage{hyperref}

%opening
\title{Supplementary methods for the manuscript "BiNoM, a Cytoscape plugin for accessing and analyzing pathways using
standard systems biology formats"}
\author{}

\begin{document}

\maketitle

%\begin{abstract}
%In this tutorial, we describe the step-by-step procedure to extract a compact modular network from a large network representing the cell cycle G1/S transition. The input file can be downloaded from our website \url{http://binom.curie.fr/}.
%\end{abstract}


\section*{Installation}

BiNoM plugin can be installed in two different ways.

\begin{itemize}
 \item Launch Cytoscape and then launch the plugin manager (menu ``Plugins $>$ Manage
Plugins"). Choose the category ``Other" in the left panel of the manager, and
select the latest version of BiNoM. Click ``Install" and restart Cytsocape. 
  \item Download the BiNoM "jar" file from our website
\url{http://binom.curie.fr} and copy it in the ``plugins" directory of the
Cytoscape installation folder. Restart Cytoscape. 
\end{itemize}


\section*{Changelog}
We provide here a list of the main changes that were introduced for the version
of BiNoM described in this manuscript, as compared to the original version by
Zinovyev et al. \cite{zinovyev2008binom}.

\begin{itemize}
 \item Support for the export and import of BioPAX level 3 files. Please note
that versions 2.x of BiNoM accepts only BioPAX level 3 files. The previous
version of BiNoM (version 1.0) is available from our website
(\url{http://binom.curie.fr}) for users that would like to process BioPAX level
2 files.

 \item Module manager: a complete set of functions that simplify the
visualization and modularization of large molecular networks. The functions rely
on a new feature introduced with Cytoscape 2.7 and higher and knwon as
\textit{nested} networks.

 \item Path Influence Quantification algorithm (PIQuant): this algorithm was
available in beta version in the previous version of BiNoM, but we provide now a
stable version with detailed explanations on the algorithm and a case study in
the manuscript.

  \item A complete case study illustrating the principal functions of BiNoM applied to the analysis of a network representing the G1/S transition of the cell cycle.

  \item Numerous bugs in the code have been fixed since the version 1.0 of BiNoM.

\end{itemize}



\section*{Tutorial for the modularization of the G1/S network using BiNoM functions}
In this tutorial, we describe the step-by-step procedure to extract a compact
modular network from a large network representing the cell cycle G1/S
transition. The input file can be downloaded from our website
\url{http://binom.curie.fr/}.
\begin{enumerate}
\item Import network. \\
\textbf{File $\rightarrow$ Import $\rightarrow$ Network (Multiple File Types)...}\\
Choose the file g1s\_209\_266.xgmml.
The network ``G1S" is created in Cytoscape.

% fournir le fichier xml correspondant

\item	Decompose the network into subnetworks. \\
We choose to decompose the network according to the different components present
in the map rather than decomposing according to the cycles of the network. 
\textbf{Plugins $\rightarrow$ BiNoM 2.1 $\rightarrow$ BiNoM Analysis $\rightarrow$ Get Material
Components…}.\\
33 subnetworks are created.

\item	Clustering. \\
The resulting subnetworks may share a lot of species. To reduce the number of
subnetworks, we cluster them with an overlap of 25\%. \\
\textbf{Plugins $\rightarrow$  BiNoM 2.1 $\rightarrow$ BiNoM Analysis $\rightarrow$ Cluster networks}. \\
A pop-up window appears. Among the proposed networks, select all networks except
G1S.xml and choose 25\% overlap. 9 networks are created.

\item	Rename the newly-created clusters.
It is important to rename right away the clusters with a name that illustrates
the content of each of these clusters. 
Choosing the appropriate name might be facilitated by listing the name of the
components of the cluster: \textbf{Plugins $\rightarrow$  BiNoM 2.1 $\rightarrow$ BiNoM module manager
$\rightarrow$ List components of species in network and modules} \\
To rename them, right-click on the Network panel.
We propose to use the following names for these 9 clusters: \textbf{E2F1\_RB},
\textbf{CycD1}, \textbf{CycE\_A}, \textbf{Wee1}, \textbf{p21}, \textbf{CycH}, \textbf{CycB1}, \textbf{CDC25} and \textbf{CDC20}.

\item	Inside the clusters, check the content of the modules (manual curation
of the modularization).

\begin{itemize}
\item	For cluster E2F1\_RB, separate RB from E2F1 and from E2F6 and create
three different modules. \textbf{Select $\rightarrow$ Nodes $\rightarrow$ By Name}. Write ``*E2F6*". The
nodes with E2F6 are highlighted.Select the reactions between the nodes.
\textbf{Select $\rightarrow$ Nodes $\rightarrow$ First neighbours of selected nodes}. \textbf{File $\rightarrow$ New
 $\rightarrow$ Network $\rightarrow$ From selected nodes all edges}. Rename ``E2F1\_RB child" to "E2F6" and delete the
nodes from E2F1\_RB network.
\item	Do the same for pRB.
\item	Rename the remaining network of E2F1\_RB, E2F1.
%\item	For cluster, CycB1, select CDC25 and create a new module with components
%including CDC25.
\item	The resulting list of modules is: CDC25, CDC20, CycB1, Wee1, CycH, p21, CycD1,
CycE\_A, E2F1, RB, E2F6.
\end{itemize}

\item	Select modules to prepare the modular view. In the pop-up window, choose
the 11 modules. \textbf{Plugins $\rightarrow$  BiNoM 2.1 $\rightarrow$ BiNoM Module Manager$\rightarrow$ Create Network of Modules}.
\item	Connect modules. \textbf{Plugins $\rightarrow$  BiNoM 2.1 $\rightarrow$ BiNoM Module Manager $\rightarrow$ Create Connections between Modules.}


\end{enumerate}
 \bibliographystyle{bmc_article}
\bibliography{binom2}
\end{document}
