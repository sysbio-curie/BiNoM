\documentclass[a4,center,fleqn]{NAR}

% Enter dates of publication
\copyrightyear{2008}
\pubdate{31 July 2009}
\pubyear{2009}
\jvolume{37}
\jissue{12}

%\articlesubtype{This is the article type (optional)}


\begin{document}

\title{NaviCell Web service for Network-based Data Visualization}


\author{%
Eric Bonnet\,$^{1,2,3}$,
Eric Viara\,$^{4}$,
Inna Kuperstein\,$^{1,2,3}$,
Laurence Calzone\,$^{1,2,3}$,
David PA Cohen\,$^{1,2,3}$,
Emmanuel Barillot\,$^{1,2,3}$,
Andrei Zinovyev\,$^{1,2,3}$%
\footnote{To whom correspondence should be addressed.
Tel: +33 (0)1 56 24 69 89; Fax: +33 (0)1 56 24 69 11; Email: andrei.zinovyev@curie.fr}}

\address{%
$^{1}$Institut Curie, 26 rue d'Ulm, 75248 Paris, France, 
$^{2}$INSERM U900, 75248 Paris, France,
$^{3}$Mines ParisTech, 77300 Fontainebleau, France,
$^{4}$Sysra, 91330 Yerres, France.
}
% Affiliation must include:
% Department name, institution name, full road and district address,
% state, Zip or postal code, country

\history{%
Received January 1, 2009;
Revised February 1, 2009;
Accepted March 1, 2009}


\bibliographystyle{nar}
\maketitle

\begin{abstract}
Text. Text. Text. Text. Text. Text. Text. Text. Text. Text. Text.
\end{abstract}


\section{Introduction}

Text. Text. Text. Text. Text. Text. Text. Text. Text. Text. Text.

Text. Text. Text. Text. Text. Text. Text. Text. Text. Text. Text.
Text. Text. Text. Text. Text. Text. Text. Text. Text. Text. Text.
Text. Text. Text. Text. Text. Text. Text. Text. Text. Text. Text.
Text. Text. Text. Text. Text. Text. Text. Text. Text. Text. Text.

Text. Text. Text. Text. Text. Text. Text. Text. Text. Text. Text.
Text. Text. Text. Text. Text. Text. Text. Text. Text. Text. Text.
Text. Text. Text. Text. Text. Text. Text. Text. Text. Text. Text.
Text. Text. Text. Text. Text. Text. Text. Text. Text. Text. Text.
Text. Text. Text. Text. Text. Text. Text. Text. Text. Text. Text.
Text. Text. Text. Text. Text. Text. Text. Text. Text. Text. Text.
Text. Text. Text. Text. Text. Text. Text. Text. Text. Text. Text.
Text. Text. Text. Text. Text. Text. Text. Text. Text. Text. Text.
Text. Text. Text. Text. Text. Text. Text. Text. Text. Text. Text.
Text. Text. Text. Text. Text. Text. Text. Text. Text. Text. Text.
Text. Text. Text. Text. Text. Text. Text. Text. Text. Text. Text.
Text. Text. Text. Text. Text. Text. Text. Text. Text. Text. Text.

% **************************************************************
% Keep this command to avoid text of first page running into the
% first page footnotes
%\enlargethispage{-65.1pt}
% **************************************************************

Text. Text. Text. Text. Text. Text.
Text. Text. Text. Text. Text. Text. Text. Text. Text. Text. Text.
Text. Text. Text. Text. Text. Text. Text. Text. Text. Text. Text.
Text. Text. Text. Text. Text. Text. Text. Text. Text. Text. Text.
Text. Text. Text. Text. Text. Text. Text. Text. Text. Text. Text.
Text. Text. Text. Text. Text. Text. Text. Text. Text. Text. Text.
Text. Text. Text. Text. Text. Text. Text. Text. Text. Text. Text.
Text. Text. Text. Text. Text. Text. Text. Text. Text. Text. Text.
Text. Text. Text. Text. Text. Text. Text. Text. Text. Text. Text.
Text. Text. Text. Text. Text. Text. Text. Text. Text. Text. Text.
Text. Text. Text. Text. Text. Text. Text. Text. Text. Text. Text.
Text. Text. Text. Text. Text. Text. Text. Text. Text. Text. Text.
Text. Text. Text. Text. Text. Text. Text. Text. Text. Text. Text.
Text. Text. Text. Text. Text. Text. Text. Text. Text. Text. Text.
Text. Text. Text. Text. Text. Text. Text. Text. Text. Text. Text.
Text. Text. Text. Text. Text. Text. Text. Text. Text. Text. Text.
Text. Text. Text. Text.
Text 

\section{ALGORITHMS AND SOFTWARE}

\subsection{Implementation}

The NaviCell software tool is implementing a set of JavaScript functions on a
web interface, allowing an easy and intuitive navigation of large-scale
molecular maps of biological networks \cite{kuperstein2013navicell}. We
conceived the NaviCell Web Service Data Visualization tool as an extension of
NaviCell allowing users to visualize and analyze different types of "omics"
data. The Web Service is implemented at two different functional levels that are
interconnected. The first level corresponds to the interactive, windows-based,
usage of the web interface to load and visualize data. This level is implemented
as a specific set of JavaScript functions linked to a button-based menu located
in the right-hand panel of the web interface. The second level corresponds to a
programmatic usage of the service, which allow users to write programs that will
communicate with the NaviCell \emph{server} in order to automate all the
visualization operations. This functional level is implemented as a RESTful web
service, using standard web protocol HTTP operations and data encoding (JSON)
to perform all the necessary operations \cite{fielding2002principled}.
RESTful web service are very popular and have the advantage of being
lightweight, efficient and simple to use. In our case, they also have the benefit of
being relatively easy to implement in many different programming language and
software packages. At the moment we have a full API ready for Python, and
upcoming APIs for R and Java. Of course, those APIs will facilitate the
integration of NaviCell services in other applications, and for instance we
have started a collaboration to offer our data visualization services through
the Garuda Alliance web platform, which aims to be a one-stop service for
bioinformatics and systems biology \cite{ghosh2011software}. The Figure 1 is
summarizing the NaviCell \emph{server} architecture and information flow.

\subsection{Input data types}

\subsection{Graphical data representation}

\section{RESULTS}

\subsection{Results subsection one}

Text. Text. Text. Text. Text. Text. Text. Text. Text. Text. Text.
Text. Text. Text. Text. Text. Text. Text. Text. Text. Text. Text.
Text. Text. Text. Text. Text. Text. Text. Text. Text. Text. Text.
Text. Text. Text. Text. Text. Text. Text. Text. Text. Text. Text.
Text. Text. Text. Text. Text. Text. Text. Text. Text. Text. Text.
Text. Text. Text. Text. Text. Text. Text. Text. Text. Text. Text.
Text. Text. Text. Text. Text. Text. Text. Text. Text. Text. Text.
Text. Text. Text. Text. Text. Text. Text. Text. Text. Text. Text.
Text. Text. Text. Text. Text. Text. Text. Text. Text. Text. Text.
Text. Text. Text. Text. Text. Text. Text. Text. Text. Text. Text.
Text. Text. Text. Text. Text. Text. Text. Text. Text. Text. Text.
Text. Text. Text. Text. Text. Text. Text. Text. Text.

\section{DISCUSSION}

\subsection{Discussion subsection one}

Text. Text. Text. Text. Text. Text. Text. Text. Text. Text. Text.

\subsection{Discussion subsection two}

Text. Text. Text. Text. Text. Text. Text. Text. Text. Text. Text.

\section{CONCLUSION}

Text. Text. Text. Text. Text. Text. Text. Text. Text. Text. Text.

\section{ACKNOWLEDGEMENTS}

Text. Text. Text. Text. Text. Text. Text. Text. Text. Text. Text.


\subsubsection{Conflict of interest statement.} None declared.
\newpage

\bibliography{biblio}

\section{FIGURE LEGENDS}

\textbf{Figure 1.} General architecture of the NaviCell Web service
\emph{server}. Client software (light blue layer) communicates with the server
(red layer) through standard HTTP requests using the standard JSON format to
encode data (RESTful web service, dark blue layer). A session (with a unique ID)
is established between the server and the client browser (yellow layer) through
Ajax communication channel to visualize the results of the commands send by the
client. It is worth noticing that communication channels are bidirectional, i.e.
the client software can send data (e.g. an expression data matrix) to the
server, but it can also receive data from the server (e.g. a list of gene HUGO
codes contained in a map).


\end{document}
