\documentclass[a4,center,fleqn]{NAR}

% Enter dates of publication
\copyrightyear{2008}
\pubdate{31 July 2009}
\pubyear{2009}
\jvolume{37}
\jissue{12}

%\articlesubtype{This is the article type (optional)}


\begin{document}

\title{NaviCell Web Service for Network-based Data Visualization}


\author{%
Eric Bonnet\,$^{1,2,3}$,
Eric Viara\,$^{4}$,
Inna Kuperstein\,$^{1,2,3}$,
Laurence Calzone\,$^{1,2,3}$,
David PA Cohen\,$^{1,2,3}$,
Emmanuel Barillot\,$^{1,2,3}$,
Andrei Zinovyev\,$^{1,2,3}$%
\footnote{To whom correspondence should be addressed.
Tel: +33 (0)1 56 24 69 89; Fax: +33 (0)1 56 24 69 11; Email: andrei.zinovyev@curie.fr}}

\address{%
$^{1}$Institut Curie, 26 rue d'Ulm, 75248 Paris, France, 
$^{2}$INSERM U900, 75248 Paris, France,
$^{3}$Mines ParisTech, 77300 Fontainebleau, France,
$^{4}$Sysra, 91330 Yerres, France.
}
% Affiliation must include:
% Department name, institution name, full road and district address,
% state, Zip or postal code, country

\history{%
Received January 1, 2009;
Revised February 1, 2009;
Accepted March 1, 2009}


\bibliographystyle{nar}
\maketitle

\begin{abstract}
Text. Text. Text. Text. Text. Text. Text. Text. Text. Text. Text.
\end{abstract}


\section{Introduction}

Text. Text. Text. Text. Text. Text. Text. Text. Text. Text. Text.

Text. Text. Text. Text. Text. Text. Text. Text. Text. Text. Text.
Text. Text. Text. Text. Text. Text. Text. Text. Text. Text. Text.
Text. Text. Text. Text. Text. Text. Text. Text. Text. Text. Text.
Text. Text. Text. Text. Text. Text. Text. Text. Text. Text. Text.

Text. Text. Text. Text. Text. Text. Text. Text. Text. Text. Text.
Text. Text. Text. Text. Text. Text. Text. Text. Text. Text. Text.
Text. Text. Text. Text. Text. Text. Text. Text. Text. Text. Text.
Text. Text. Text. Text. Text. Text. Text. Text. Text. Text. Text.
Text. Text. Text. Text. Text. Text. Text. Text. Text. Text. Text.
Text. Text. Text. Text. Text. Text. Text. Text. Text. Text. Text.
Text. Text. Text. Text. Text. Text. Text. Text. Text. Text. Text.
Text. Text. Text. Text. Text. Text. Text. Text. Text. Text. Text.
Text. Text. Text. Text. Text. Text. Text. Text. Text. Text. Text.
Text. Text. Text. Text. Text. Text. Text. Text. Text. Text. Text.
Text. Text. Text. Text. Text. Text. Text. Text. Text. Text. Text.
Text. Text. Text. Text. Text. Text. Text. Text. Text. Text. Text.

Text. Text. Text. Text. Text. Text.
Text. Text. Text. Text. Text. Text. Text. Text. Text. Text. Text.
Text. Text. Text. Text. Text. Text. Text. Text. Text. Text. Text.
Text. Text. Text. Text. Text. Text. Text. Text. Text. Text. Text.
Text. Text. Text. Text. Text. Text. Text. Text. Text. Text. Text.
Text. Text. Text. Text. Text. Text. Text. Text. Text. Text. Text.
Text. Text. Text. Text. Text. Text. Text. Text. Text. Text. Text.
Text. Text. Text. Text. Text. Text. Text. Text. Text. Text. Text.
Text. Text. Text. Text. Text. Text. Text. Text. Text. Text. Text.
Text. Text. Text. Text. Text. Text. Text. Text. Text. Text. Text.
Text. Text. Text. Text. Text. Text. Text. Text. Text. Text. Text.
Text. Text. Text. Text. Text. Text. Text. Text. Text. Text. Text.
Text. Text. Text. Text. Text. Text. Text. Text. Text. Text. Text.
Text. Text. Text. Text. Text. Text. Text. Text. Text. Text. Text.
Text. Text. Text. Text. Text. Text. Text. Text. Text. Text. Text.
Text. Text. Text. Text. Text. Text. Text. Text. Text. Text. Text.
Text. Text. Text. Text.
Text 

\section{ALGORITHMS AND SOFTWARE}

\subsection{Implementation}

The NaviCell software tool implements a set of JavaScript functions on a
web interface, allowing an easy and intuitive navigation of large-scale
molecular maps of biological networks \cite{kuperstein2013navicell}. We
conceived the NaviCell Web Service Data Visualization tool as an extension of
NaviCell allowing users to visualize and analyze different types of "omics"
data. The Web Service is implemented at two different functional levels that are
interconnected. The first level corresponds to the interactive, windows-based,
usage of the web interface to load and visualize data. This level is implemented
as a specific set of JavaScript functions linked to a button-based menu located
in the right-hand panel of the web interface. The second level corresponds to a
programmatic usage of the service, which allow users to write programs that will
communicate with the NaviCell \emph{server} in order to automate all the
visualization operations. This functional level is implemented as a RESTful web
service, using standard web protocol HTTP operations and data encoding (JSON)
to perform all the necessary operations \cite{fielding2002principled}.
RESTful web service are very popular and have the advantage of being
lightweight, efficient and simple to use. In our case, they also have the benefit of
being relatively easy to implement in many different programming language and
software packages. At the moment we have a full API ready for Python, and
upcoming APIs for R and Java. Of course, those APIs will facilitate the
integration of NaviCell services in other applications, and for instance we
have started a collaboration to offer our data visualization services through
the Garuda Alliance web platform, which aims to be a one-stop service for
bioinformatics and systems biology \cite{ghosh2011software}. The Figure 1 is
summarizing the NaviCell \emph{server} architecture and information flow.

% **************************************************************
% Keep this command to avoid text of first page running into the
% first page footnotes
\enlargethispage{-65.1pt}
% **************************************************************


\subsection{Input data types}

We have designed the NaviCell data visualization service to be able to accept
several types of "omics" data. The complete list of data types that are
currently accepted for input is provided in Table \ref{table:01}. The different
types are mapped to internal representations that determine what kind of
treatment will be done and what type of options will be possible to visualize
the data.

For instance, a mRNA expression data matrix will be associated to a
"continuous" internal representation. Thus, if the user is choosing to display
this data with a heatmap, a color gradient will be chosen by default, with the
appropriate options for modifying this representation. On the other hand, when a
matrix with discrete copy-number data is loaded, it is associated with a
"discrete unordered" internal representation, and this time a color palette will
be used for visualization, with a distinct color associated to each copy-number
value.

The input format for data sets is standard tab-delimited text files, with rows
representing genes and columns representing samples (or experiments, or time
points). Genes in the first column should be represented by their HUGO (HGNC)
gene symbols, that will be associated to the different entities (genes,
proteins, complexes) of the map.  

Users can also load annotation files to specify how samples can be grouped (e.g.
disease vs control, as a simple tab-delimited text file with sample names as
rows and grouping factors as columns). Then, an appropriate method will be used
to summarize the values of all the samples contained in a group, defined by the
internal representation. For instance, expression values for a group of samples
will be averaged by default for the "continuous" type, and the user will have
the option to change the defult method and use the median, the minimum value,
the maximum value, etc. For mutation data ("discrete unordered" type), taking
the average does not make sense and instead the grouping method will default to
"at least one element of the group is mutated". 

%\begin{table}[b]
\begin{table}
\tableparts{%
\caption{List of current input data types for NaviCell data visualization. The
first column list the types as they appear in the NaviCell interface. The
second column list the internal representations that are used to determine what
type of data treatment will be applied for visualization.}
\label{table:01}%
}{%
\begin{tabular*}{\columnwidth}{@{}ll@{}}
\toprule
"Omics" data type &  Internal representation
%\\
%& (\%) & (s$^{-1}$) & (\%) & (s$^{-1}$)
\\
\colrule
mRNA expression data & continuous
\\
microRNA expression data & continuous
\\
Protein expression data & continuous
\\
Discrete copy number data & discrete ordered 
\\
Continuous copy number data & continuous
\\
Mutation data & discrete unordered 
\\
Gene list & set 
\\
\botrule
\end{tabular*}%
}
{
}
\end{table}

\subsection{Graphical data representations}

Biology is a scientific discipline deeply grounded in visual representations,
that are important vectors to communicate results and ideas. Moreover, there is
a strong incentive to provide visual representations that are
\emph{interactive}, particularly on web-sites, where users can easily play with
different options to tweak and adapt the representation to their needs.

We have included in the NaviCell web service for data visualization a number of
different graphical representations. Some of them are rather standard
and broadly used in biology (heatmaps, barplots), while others (glyphs, map
staining) have not been previously employed (to our knowledge) in the context of network-based
visualization. 

\begin{itemize}

\item \textbf{Simple markers} are pictograms (similar to the ones found in Goggle Maps
to indicate geographical locations) that are draw on the map at the location
of a given species (protein, gene, complex). They are used to display the
results of a search done by the user or to display a list of genes of interest
uploaded on the map.

\item \textbf{Heatmaps} display individual values of a matrix as
colors. They are very often used in biology, especially to display expression
data measured for example from microarray experiments. In NaviCell, heatmaps can be used to
display continuous data types such as expression values, or discrete data such
as copy-number or mutation values. The user can arrange the display to have
several samples or group of samples displayed, and/or different experiments. 

\item \textbf{Barplots} are charts with rectangular bars proportional to the values
they represent. On our web interface, they are also colored according to the
data value. Barplots can be very efficient to visually distinguish different
values between two or more groups of conditions (e.g. disease/control).  

\item \textbf{Glyphs} are graphical representation using basic geometrical
shapes (triangle, square, rectangle, diamond, hexagon). The user can specify
which datasets to use for the shape, the size and the color of the glyph. This
type of representation is particularly useful to visually combine different
types of data. For example, one might consider using matched data for
copy-number and expression on a set of samples. In this case, the shape of the
glyphs could be assigned to the copy-number values, while the expression data
would be used for the color of the glyph. Like this, users can quickly
appreciate the two data types values at the same time on the glyph.

\item \textbf{Map staining} is a novel data representation where pre-defined
areas (or territories) around each entity are colored according to the data
value. The areas are defined on the map using Voronoi's diagrams. This
geometrical algorithm divides a surface into a finite set of convex polygons
\cite{aurenhammer1991voronoi}, using a set of points (seeds) to define the
polygons' shape. In our case, we use all the entities present on a molecular map
as the seeds. The polygons are pre-calculated for each map by an algorithm, and
are colored dynamically on the web interface according to the values of the data
sets and the options defined by the user.  

\end{itemize}

\section{RESULTS}

Text. Text. Text. Text. Text. Text. Text. Text. Text. Text. Text.
Text. Text. Text. Text. Text. Text. Text. Text. Text. Text. Text.
Text. Text. Text. Text. Text. Text. Text. Text. Text. Text. Text.
Text. Text. Text. Text. Text. Text. Text. Text. Text. Text. Text.
Text. Text. Text. Text. Text. Text. Text. Text. Text. Text. Text.
Text. Text. Text. Text. Text. Text. Text. Text. Text. Text. Text.
Text. Text. Text. Text. Text. Text. Text. Text. Text. Text. Text.
Text. Text. Text. Text. Text. Text. Text. Text. Text. Text. Text.
Text. Text. Text. Text. Text. Text. Text. Text. Text. Text. Text.
Text. Text. Text. Text. Text. Text. Text. Text. Text. Text. Text.
Text. Text. Text. Text. Text. Text. Text. Text. Text. Text. Text.
Text. Text. Text. Text. Text. Text. Text. Text. Text.

\section{DISCUSSION}

We have compared the features of the NaviCell Web Service for data visualization
with similar tools. We selected a number of web-based tools that are providing
similar functionalities, i.e. easy pathway browsing functions and interactive data
visualization capabilities. For the comparison, we have focused on the features
related to map navigation, molecular data types, graphical representations,
and programmatic access. Table \ref{table:02} display the results. NaviCell Web
Service offers more features than other tools in terms of map navigation, for
data visualization with additional graphical representations (such as map
staining) and more data types, and finally extended support for programmatic
access from different computer languages.        


%\begin{table}[b]
\begin{table}
\tableparts{%
\caption{Comparison of the NaviCell Web Service features with similar web sites for pathway-based data visualization.
}
\label{table:02}%
}{%
\begin{tabular*}{\columnwidth}{@{}lcccccc@{}}
\toprule
 Features & Na & Re & KE & iP & Bc & Pa
%\\
%& (\%) & (s$^{-1}$) & (\%) & (s$^{-1}$)
\\
\colrule
Map: navigation & $\bullet$ & $\bullet$ & & $\bullet$ & $\bullet$ & $\bullet$ 
\\
Map: simple zooming & $\bullet$ & $\bullet$ & $\bullet$ & $\bullet$ & $\bullet$ &  $\bullet$ 
\\
Map: semantic zooming & $\bullet$ & & & & &
\\
Visualization: node coloring & & $\bullet$ & & & & $\bullet$ 
\\
Visualization: heatmaps & $\bullet$ & & & & $\bullet$ &
\\
Visualization: barplots & $\bullet$ & & & & $\bullet$ &
\\
Visualization: glyphs & $\bullet$ & & & & &
\\
Visualization: map staining & $\bullet$ & & & & &
\\
Data mapping: gene lists& $\bullet$ & $\bullet$ & $\bullet$ & $\bullet$ & $\bullet$ & $\bullet$ 
\\
Data mapping: expression data& $\bullet$ & $\bullet$ & & & $\bullet$ & $\bullet$ 
\\
Data mapping: copy-number data& $\bullet$ & & & & &
\\
Data mapping: mutation data& $\bullet$ & & & & &
\\
Data mapping: metabolomic data& & $\bullet$ & & & &
\\
Data mapping: interactions & & $\bullet$ & & & &
\\
Programmatic access: RESTful web & $\bullet$ & $\bullet$ & $\bullet$ & & &
\\
Programmatic access: data visual. & $\bullet$ & & & & &
\\
\botrule
\end{tabular*}%
}
{Abbreviations: Na: NaviCell Web Service, Re: Reactome\cite{croft2010reactome},
KE: KEGG\cite{kanehisa2000kegg}, iP: iPath\cite{letunic2008ipath}, Bc:
BioCyc\cite{karp2005expansion}, Pa: PATIKAweb\cite{demir2002patika}.
}
\end{table}


\section{CONCLUSION}

Text. Text. Text. Text. Text. Text. Text. Text. Text. Text. Text.

\section{ACKNOWLEDGEMENTS}

Text. Text. Text. Text. Text. Text. Text. Text. Text. Text. Text.


\subsubsection{Conflict of interest statement.} None declared.
\newpage

\bibliography{biblio}

\section{FIGURE LEGENDS}

\textbf{Figure 1.} General architecture of the NaviCell Web service
\emph{server}. Client software (light blue layer) communicates with the server
(red layer) through standard HTTP requests using the standard JSON format to
encode data (RESTful web service, dark blue layer). A session (with a unique ID)
is established between the server and the client browser (yellow layer) through
Ajax communication channel to visualize the results of the commands send by the
client. It is worth noticing that communication channels are bidirectional, i.e.
the client software can send data (e.g. an expression data matrix) to the
server, but it can also receive data from the server (e.g. a list of gene HUGO
codes contained in a map).


\end{document}
