\documentclass[11pt]{article} 
\usepackage{geometry}
\geometry{a4paper}
\usepackage{graphicx}
\usepackage{hyperref}
\usepackage{amsmath}


\begin{document}

\begin{titlepage}
 
\vspace*{\fill}
\begin{figure}[h]
 \center{\includegraphics[width=0.9\textwidth]{graphics/title.png}}
\end{figure}
\vspace*{\fill}

\newpage

\title{\huge{\textbf{BiNoM: A Biological Network Manager}}}
\date{}
\author{}
\maketitle

\vspace{4 cm}
\center{\textbf{\huge{Manual}}} 
\vspace{4 cm}
\center{\small{Eric Bonnet, Laurence Calzone, Daniel Rovera, Gautier Stoll, Emmanuel Barillot, Andrei Zinovyev}}

\center{Team Computational Systems Biology of Cancer \\ Institut Curie \\ 26, rue d'Ulm \\ 75005 Paris}

\end{titlepage}

\tableofcontents

\newpage

\section{Introduction}
BiNoM (BIological NetwOrk Manager) is a Cytoscape plugin, developed to
facilitate the manipulation of biological networks represented in standard
systems biology formats and to carry out studies on the network structure. BiNoM
provides the user with a complete interface for the analysis of biological
networks in Cytoscape environment.\\

In an effort to exchange and curate pathway database knowledge, several standard
formats have been developed (SBML, BioPAX \cite{stromback2005representation} and
others). Many softwares, which are centered on the description and
representation of biological pathways, adopted these standards.
CellDesigner\cite{kitano2005using} and Cytoscape\cite{shannon2003cytoscape}, for
instance, allow the visualization and manipulation of networks but meet some
limitations. BiNoM was designed to facilitate the use of systems biology
standards, the extraction and organization of information from pathway databases
through BioPAX interface.\\

BiNoM concentrates on the following aspects: the import and export of BioPAX and
(CellDesigner) SBML files and the conversion between them; the structural
analysis of biological networks including decomposition of networks into
modules, path analysis, etc.; the BioPAX query engine which provides the
extraction of information from huge BioPAX files such as whole pathway
databases; and various operations on graphs not offered by Cytoscape such as
clipboard operations and comparison of networks.\\

BiNoM plugin with documentation, API and source code is available for download (\url{http://binom.curie.fr}).\\

\section{BiNoM I/O}
\subsection{Import BioPAX 3 document}
BioPAX level 3 information is fully supported (reaction network, interaction network, pathway structure, annotations).\\\\
\textbf{Plugins$\Rightarrow$BiNoM 2.1$\Rightarrow$BiNoM  I/O$\Rightarrow$Import BioPAX 3 Document from file}\\
The model M-Phase-L3.owl \cite{novak1998model} is uploaded. A dialog window
proposes to create three different interfaces from the BioPAX file: reaction
network (RN), pathway structure (PS) and interaction map (IM).

\begin{itemize}
\item Reaction network: M-Phase-L3 RN is a representation of the reaction network (figure~\ref{View_BioPAX_of_Novak}).
\item Pathway structure: M-Phase-L3 PS represents the pathway hierarchical
structure. For this example, we choose to show a more detailed and complete
pathway, the apoptosis sub-network extracted from Reactome database
(figure~\ref{Apoptosis_pathway_hierarchical_structure}).
\item Protein interaction: M-Phase-L3 IM shows which proteins interact with each other.
\end{itemize}
For more details on BioPAX, its interfaces, etc, go to section \ref{Standard_BioPAX_Interfaces}.
\begin{figure}
\centering
\includegraphics[width=0.8\textwidth]{graphics/View_BioPAX_of_Novak}
\caption{BioPAX view of Novak et al. model.}
\label{View_BioPAX_of_Novak}
\end{figure}
\begin{figure}
\centering
\includegraphics[width=0.8\textwidth]{graphics/Apoptosis_pathway_hierarchical_structure}
%\caption{Apoptosis pathway hierarchical structure. Green nodes represent pathways, pink triangular nodes represent steps, and grey nodes represent reactions. From the apoptosis node (top node in red), the cell can choose through 5 different paths. The red-colored path shows one of them, the activation of apoptosis via the intrinsic pathway, leading to the cleavage of caspases 3.}
\caption{Apoptosis pathway hierarchical structure. Green nodes represent
pathways, pink triangular nodes represent steps, and grey nodes represent
reactions.}
\label{Apoptosis_pathway_hierarchical_structure}
\end{figure}
\\In the case of creating the pathway structure interface, several choices are offered:
\begin{itemize}
\item Make Root Pathway Node: adds an extra node to which all pathways are connected. This feature can be useful for organizing the graph and joining separate and disjoint pathways.
\item Include Next Links: shows the ‘order’ of the reactions. From a node, an arrow indicates which node is the next step. This feature provides a timeline of the events in a pathway and could emphasize, for example, the linearity of a cascade.
\item Include Pathways: includes green nodes (figure~\ref{Apoptosis_pathway_hierarchical_structure}) which correspond to the names of the different pathways of the network.
\item Include interactions: shows explicitly the reactions involved in the pathway (lower grey nodes in figure~\ref{Apoptosis_pathway_hierarchical_structure}).
\end{itemize}\
\textbf{Plugins$\Rightarrow$BiNoM 2.1$\Rightarrow$BiNoM I/O$\Rightarrow$Import BioPAX 3 Document from URL}\\
A BioPAX 3 document can also be imported directly from a URL. The web address must be typed in the dialog window.



\subsection{Import CellDesigner document}
\textbf{Plugins$\Rightarrow$BiNoM 2.1$\Rightarrow$BiNoM I/O$\Rightarrow$Import CellDesigner Document from file}\\
The model can be drawn – or downloaded\cite{novak1998model} – in CellDesigner (figure~\ref{CellDesigner_view_of_yeast_cell_division}) and saved as M-Phase.xml.\\
The “Import CellDesigner Document from file” function imports a model from CellDesigner to Cytoscape.  A dialog window opens and M-Phase.xml needs to be selected and imported (figure~\ref{Cytoscape_view_of_yeast_cell_division}).\\
\begin{figure}
\centering
\includegraphics[width=0.8\textwidth]{graphics/CellDesigner_view_of_yeast_cell_division}
\caption{CellDesigner view of the cell division cycle model of fission yeast\cite{novak1998model}}
\label{CellDesigner_view_of_yeast_cell_division}
\end{figure}\begin{figure}
\centering
\includegraphics[width=0.8\textwidth]{graphics/Cytoscape_view_of_yeast_cell_division}
\caption{Cytoscape view of the cell division cycle model of fission yeast from a CellDesigner document}
\label{Cytoscape_view_of_yeast_cell_division}
\end{figure}
\\Figure~\ref{CellDesigner_view_of_yeast_cell_division}~and~\ref{Cytoscape_view_of_yeast_cell_division} show the same model viewed by CellDesigner and Cytoscape respectively. The layout information from CellDesigner is imported automatically into Cytoscape.\\\\
In species notes in CellDesigner “Attribute name:Value” as HUGO:E2F1 (without blank) is converted in Cytoscape as the attribute HUGO with the value E2F1 for the specie.\\\\



\subsection{Import CSML document}
\textbf{Plugins$\Rightarrow$BiNoM 2.1$\Rightarrow$BiNoM I/O$\Rightarrow$Import CSML document}\\
BiNoM imports a CSML (Cell System Markup Language, csml.org)



\subsection{Import influence network from AIN file} \label{Import_AIN_file}

\textbf{Plugins$\Rightarrow$BiNoM 2.1$\Rightarrow$BiNoM I/O$\Rightarrow$Import AIN file}\\

This option proposes to automatically import an influence network in Cytoscape
from a simple tab-delimited text file. We call the format of the
text file AIN (Annotated Influence Network). Basically each row of the text file is encoding an
interaction between two species, with a few optional descriptive fields. The precise
format for each field is described in the appendices (section~\ref{AIN_file_format}). 

The AIN file of the apoptosis model ExamplApop.txt is imported. First, the user
is asked to manage the families (groups of genes or proteins, see the appendices
for a precise description): they can be expanded (replacing the family by all
its members) or collapsed (replacing all family members by the name of the
family). See figure~\ref{AIN_dialog_for_families_management}.\\\\

\begin{figure}
\centering
\includegraphics[width=0.8\textwidth]{graphics/AIN_dialog_for_families_management}
\caption{Dialog window for families management when importing the apoptosis network influence file (AIN format). BiNoM automatically detects gene families and proposes to either collapse or expand them.}
\label{AIN_dialog_for_families_management}
\end{figure}

Then a dialog window proposes to add constitutive reactions: influences that
link proteins (or families) to their complexes and proteins (or families) to
their phosphorylated state. See
figure~\ref{AIN_dialog_for_constitutive_reactions}.\\\\

\begin{figure}
\centering
\includegraphics[width=0.5\textwidth]{graphics/AIN_dialog_for_constitutive_reactions}
\caption{Dialog window for constitutive reactions when importing apoptosis influence file. BiNoM detects all possible constitutive reactions and proposes to add them.}
\label{AIN_dialog_for_constitutive_reactions}
\end{figure}
The imported network is synchronized with BioPAX format that includes the annotations of the AIN file. All this information can be accessed via “BioPAX 3 property editor” (see BioPAX Utils, section ~\ref{BioPAX_Property_Editor}).



\subsection{Import reaction network from BiNoM reaction format file} \label{Import_reaction_format}
This function is available in the experimental version of BiNoM. 
A network can be written as a list of reactions. The network can be a biochemical reaction network or an influence network following the notation below.

\includegraphics[width=1\textwidth]{graphics/syntax_text_network.png} 

Here we give an example of an influence network: 
\begin{verbatim}
MOMP-+>CYTOCHROME_C
APAF1-+>CASP9
SMAC-|>XIAP
FADD-+>CASP8
CASP8-+>CASP3
CASP8-+>BID
BAX-|>MOMP
CASP9-+>CAPS3
XIAP-|>CASP3
XIAP-|>CASP9
BID-+>MOMP
\end{verbatim}

The file must be saved in a txt format (with extension .txt). Here, the file is saved: apoptosis.txt

A network written with the BiNoM reaction format can then be imported in Cytoscape. \\
\textbf{Plugins$\Rightarrow$BiNoM 2.5$\Rightarrow$BiNoM I/O$\Rightarrow$Import reaction network from BiNoM reaction format file}\\

Choose apoptosis.txt

\begin{figure}
\centering
\includegraphics[width=1\textwidth]{graphics/Import_brff.png} 
\caption{Network written with BiNoM format file and imported in Cytoscape. No layout is applied here.}
\label{Network_format_file}
\end{figure}



\subsection{Export current network to BioPAX 3 or to CellDesigner} \label{Export_current_network}
The Cytoscape networks can be exported in BioPAX and CellDesigner by:
\begin{itemize}
\item \textbf{Plugins$\Rightarrow$BiNoM 2.1$\Rightarrow$BiNoM I/O$\Rightarrow$Export current network to BioPAX 3}
\item \textbf{Plugins$\Rightarrow$BiNoM 2.1$\Rightarrow$BiNoM I/O$\Rightarrow$Export current network to CellDesigner}
\end{itemize}
\includegraphics[width=20pt,height=20pt]{graphics/warning} The network to be exported should be associated to an existing CellDesigner or BioPAX file by using the functions:
\begin{itemize}
\item Associate BioPAX Source, see section~\ref{Associate_BioPAX_Source}.
\item Associate CellDesigner Source,see~\ref{Associate_CellDesigner_Source}.
\end{itemize}
BiNoM is able to convert CellDesigner to BioPAX, and BioPAX
reaction network interface to pure SBML. BiNoM can also export only a
part of a CellDesigner and BioPAX file, visible in the current Cytoscape network
(interface). During the export operation, BiNoM can merge a part of
associated BioPAX file with another part that has been saved already. BiNoM can modify the
content of a BioPAX file. \\\\

\includegraphics[width=20pt,height=20pt]{graphics/warning} BiNoM is NOT able to
create a CellDesigner file with all the graphical notations from a BioPAX file or from
scratch, and it is also not able to modify the content of a CellDesigner file.\\\\

Here are a couple of typical scenarios where BiNoM export operations can be useful.
\begin{enumerate}
\item User imports a big BioPAX file as reaction network and using Cytoscape creates a new subnetwork from the global reaction graph. After he can export this subnetwork into a separate self-containing BioPAX file.
\item User imports the pathway structure of a big BioPAX file and selects only a few pathway or pathwayStep nodes he is interested in. After he can export a part of the BioPAX file necessary to define these pathways.
\item User imports a BioPAX file as reaction network, selects a subnetwork and exports it as pure SBML to be used for creation of a computational model of this subpart later.
\item User imports CellDesigner file, selects a subnetwork and exports it as a CellDesigner file: it can be useful for creating a CellDesigner image of a network module of a big reaction network.
\item User imports CellDesigner file, selects a subnetwork and exports it as a BioPAX file (some SBML-specific information such as parameters values will be lost).
\end{enumerate}
The networks created as a result of the import operation are already associated to the corresponding BioPAX or CellDesigner files. However, if the XGMML file is saved and used in another Cytoscape session, or if a new network is created from the initial network with Cytoscape New menu then this association is lost.\\\\
To perform export operation, the network should be Re-associated to the corresponding file (from which it is originated) through Plugins$\Rightarrow$BiNoM 2.1$\Rightarrow$BiNoM I/O$\Rightarrow$Associate$\ldots$ operation. For huge BioPAX files the association might take some time for the first association, but once the file is loaded into memory cache, the following associations are almost instantaneous.\\\\
To understand better what BiNoM can do or can not, read the sections \ref{Attributed_graph_model}, \ref{BiNoM_Naming_Service} and \ref{Standard_BioPAX_Interfaces} about the BiNoM data model.



\subsection{Create CellDesigner file from current network}
This function is available in the experimental version of BiNoM. \\
\textbf{Plugins$\Rightarrow$BiNoM 2.5$\Rightarrow$BiNoM I/O$\Rightarrow$Create CellDesigner file from current network}\\

The file needs to be saved with the extension .xml.


This function is different than:\\
\textbf{Plugins$\Rightarrow$BiNoM 2.5$\Rightarrow$BiNoM I/O$\Rightarrow$Export current network to CellDesigner}\\
With this export function, the user needs to associate the network to an existing CellDesigner file. It is useful when creating a subnetwork from an existing CellDesigner network, for instance.

Here we save our network with the name: apoptosis.xml.

Open the saved file into CellDesigner. Apply the orthogonal layout or re-arrange the network as you wish.


\begin{figure}
\includegraphics{graphics/Import_CD_brff.png}
\caption{Network written with BiNoM format file and imported in CellDesigner. Orthogonal layout is applied here.}
\label{import_CD}
\end{figure}

The network exported in CellDesigner can also be a network imported from a BioPAX file. This function allows the visualization of a BioPAX file in CellDesigner.  For that, you need to import a BioPAX file using the function:\\
\textbf{Plugins$\Rightarrow$BiNoM 2.5$\Rightarrow$BiNoM I/O$\Rightarrow$Import reaction network from BioPAX}\\


\begin{itemize}
\item Importing any type of BioPAX file 

Here we choose exampleBIOPAX.owl.

\includegraphics{graphics/example_BioPAX.png} 

Then, export the current network to CellDesigner using the function:\\
\textbf{Plugins$\Rightarrow$BiNoM 2.5$\Rightarrow$BiNoM I/O$\Rightarrow$Create CellDesigner file from current network}\\

Save the file as: example.BIOPAX.xml

Open the file in CellDesigner:\\

\includegraphics{graphics/BioPAX_to_CD.png} 

\item Importing a BioPAX file from WikiPathways \\
Here are the steps: 
\begin{enumerate}
\item Download the pathway $Senescence and Autophagy (Homo sapiens)$ and save the file $WP615_60842 RN$ from WikiPathways website; 
\item Import network in Cytoscape with the function: 
\textbf{Plugins$\Rightarrow$BiNoM 2.5$\Rightarrow$BiNoM I/O$\Rightarrow$Import reaction network from BioPAX}
\item Create a CellDesigner file from this network:
\textbf{Plugins$\Rightarrow$BiNoM 2.5$\Rightarrow$BiNoM I/O$\Rightarrow$Create CellDesigner file from current network}\\
And save it as: $WP615_60842.xml$.
\item Launch CellDesigner and open the file: $WP615_60842.xml$
\begin{figure}
\includegraphics[width=1\textwidth]{graphics/wikipathways_CD.png}
\caption{Network downloaded in WikiPathways and imported in CellDesigner. Orthogonal layout is applied here.}
\label{import_CD}
\end{figure}
\item Apply a layout (here organic).
\end{enumerate}

\end{itemize}

\subsection{Export current network to SBML}
\textbf{Plugins$\Rightarrow$BiNoM 2.1$\Rightarrow$BiNoM I/O$\Rightarrow$Export current network to SBML}\\
Export the current network to pure SBML level 2.


\subsection{Export network to BiNoM reaction format}
\textbf{Plugins$\Rightarrow$BiNoM 2.1$\Rightarrow$BiNoM I/O$\Rightarrow$Export network to BiNoM reaction format}\\

This function allows to translate a network into a list of reactions. The file is saved as a text file.

As an example, we imported the file M-Phase.xml and exported the network to BiNoM reaction format file. The file is saved as a txt file (MPhasebrff.txt) as follows: 

\begin{verbatim}
(APC':Slp1')||active'|active
(APC:Slp1)||active
Cdc13:Cdc2|Thr167_pho:Rum1
Rum1|pho
Rum1
Cdc13|ubi
(Cdc13:Cdc2|Thr167_pho)||active
Cdc13:Cdc2|Thr14_pho|Tyr15_pho|Thr167_pho
Cdc13:Cdc2|Thr14_pho|Tyr15_pho
Cdc13:Cdc2|Tyr15_pho
Cdc13:Cdc2
Cdc13
Plo1
Cdc25|pho|active
Wee1|pho
Wee1
Cdc2|Thr167_pho
Cdc25
Cdc2
PP2A
CAK

Rum1+(Cdc13:Cdc2|Thr167_pho)||active-:>Cdc13:Cdc2|Thr167_pho:Rum1
Rum1|pho->null
Rum1-(Cdc13:Cdc2|Thr167_pho)||active->Rum1|pho
(APC:Slp1)||active-(Cdc13:Cdc2|Thr167_pho)||active->(APC':Slp1')||active'|active
Cdc13+Cdc2-:>Cdc13:Cdc2
Cdc25|pho|active-PP2A->Cdc25
Cdc25-(Cdc13:Cdc2|Thr167_pho)||active-Plo1->Cdc25|pho|active
Wee1-(Cdc13:Cdc2|Thr167_pho)||active->Wee1|pho
Cdc13|ubi->null
(Cdc13:Cdc2|Thr167_pho)||active-(APC':Slp1')||active'|active-=>Cdc13|ubi+Cdc2|Thr167_pho
Cdc2|Thr167_pho->Cdc2
Cdc13:Cdc2|Thr14_pho|Tyr15_pho|Thr167_pho-Cdc25|pho|active->(Cdc13:Cdc2|Thr167_pho)||active
Cdc13:Cdc2|Thr14_pho|Tyr15_pho-CAK->Cdc13:Cdc2|Thr14_pho|Tyr15_pho|Thr167_pho
Cdc13:Cdc2|Tyr15_pho-Wee1->Cdc13:Cdc2|Thr14_pho|Tyr15_pho
Cdc13:Cdc2-Wee1->Cdc13:Cdc2|Tyr15_pho
\end{verbatim}

Note that BiNoM lists first all the species and then the reactions. 


We propose three scenarios that use BiNoM reaction format file.
\begin{itemize}
\item First scenario: create a CellDesigner file from a textual model
	\begin{enumerate}
		\item Write network in text format:
			\begin{verbatim}
			A => B
			B+C=>D+E
			...
			\end{verbatim}
		\item Import reaction network in Cytoscape using BiNoM
		\item Create CellDesigner file from current network
		\item Open CellDesigner and import the newly saved file
	\end{enumerate}
\item Second scenario: create a CellDesigner file from BioPAX file
	\begin{enumerate}
		\item Import a BioPAX file in Cytoscape
		\item Create CellDesigner file from current network
		\item Open CellDesigner and import the newly saved file
	\end{enumerate}
\end{itemize}


\subsection{Associate BioPAX 3 Source} \label{Associate_BioPAX_Source}
\textbf{Plugins$\Rightarrow$BiNoM 2.1$\Rightarrow$BiNoM I/O$\Rightarrow$Associate BioPAX 3 Source}\\
Associate a BioPAX 3 Source to allow exportation in BioPAX 3 as explained in section~\ref{Export_current_network}




\subsection{Save whole associated BioPAX 3 as}
When the content of the BioPAX file is modified (through the BioPAX property editor, see section ~\ref{BioPAX_Property_Editor}), it can be saved as a whole (not only the visible part) by\\
\textbf{Plugins$\Rightarrow$BiNoM 2.1$\Rightarrow$BiNoM I/O$\Rightarrow$Save whole associated BioPAX 3 as}\\
Otherwise, all modifications made in the different interfaces are lost. Changes are visible but only recorded permanently when the document is saved to a file.



\subsection{Associate CellDesigner Source}\label{Associate_CellDesigner_Source}
\textbf{Plugins$\Rightarrow$BiNoM 2.1$\Rightarrow$BiNoM I/O$\Rightarrow$Associate BioPAX 3 Source} \\
Associate a CellDesigner Source to allow exportation in CellDesigner as explained in section~\ref{Export_current_network}



\subsection{List all reactions}
\textbf{Plugins$\Rightarrow$BiNoM 2.1$\Rightarrow$BiNoM I/O$\Rightarrow$List all reactions} \\
Display list of reactions, can be copied by control+A then control+C.



\subsection{List all nodes}
\textbf{Plugins$\Rightarrow$BiNoM 2.1$\Rightarrow$BiNoM I/O$\Rightarrow$List all nodes} \\
Display list of nodes, can be copied by control+A then control+C.



\subsection{Color CellDesigner proteins}
\textbf{Plugins$\Rightarrow$BiNoM 2.1$\Rightarrow$BiNoM I/O$\Rightarrow$Color CellDesigner proteins}\\
Cytoscape allows coloring nodes according to values of attributes (for example expression data) by the powerful possibilities of VizMapper. The export to CellDesigner keeps the colors. This process can be used to color species in CellDesigner. The function Color CellDesigner proteins allows to color proteins in CellDesigner which describe the components of complexes.\\\\
The gene expression file is based on Hugo names, data in columns (first line title and tabulation as column separator):\\Hugo names\textless Tab\textgreater expression level 1\textless Tab\textgreater expression level 2$\ldots$\\\\
Open dialog box “Color CellDesigner proteins…”, input CellDesigner file name and gene expression file, click on ok. BiNoM generate a file *.conv where Hugo names are converted in protein names (links by annotation in CellDesigner, check if correct) and a CellDesigner file by column. When there are several Hugo name the highest is kept.\\\\
Figure~\ref{Colored_CellDesigner_view_by_ficticious_data} shows the aspect of colored proteins inside complexes.
\begin{figure}
\centering
\includegraphics[width=0.8\textwidth]{graphics/Colored_CellDesigner_view_by_ficticious_data}
\caption{CellDesigner view of an extract from Rb-E2F\cite{calzone2008comprehensive} pathway colored by ficticious expression data}
\label{Colored_CellDesigner_view_by_ficticious_data}
\end{figure}



\subsection{Modify CellDesigner notes}
\textbf{Plugins$\Rightarrow$BiNoM 2.1$\Rightarrow$BiNoM I/O$\Rightarrow$Color CellDesigner proteins}\\
Modify in Cytoscape the notes of CellDesigner file when exporting.

\clearpage

\section{BiNoM Analysis}
We illustrate, here, the different functions of BiNoM related to the structural analysis, using the modified version of the Novak et al. model, M-Phase.xml as an example (figure~\ref{Cytoscape_view_of_the_M-Phase_network}).
\begin{figure}
\centering
\includegraphics[width=0.8\textwidth]{graphics/Cytoscape_view_of_the_M-Phase_network.png}
\caption{Cytoscape view of the M-Phase network}
\label{Cytoscape_view_of_the_M-Phase_network}
\end{figure}
\\From the menu Plugins$\Rightarrow$BiNoM 2.1$\Rightarrow$ BiNoM analysis, we review all the functions one by one.

\subsection{Get connected components}
\textbf{Plugins$\Rightarrow$BiNoM 2.1$\Rightarrow$BiNoM Analysis$\Rightarrow$Get connected components}\\
This command dissociates the unconnected subparts of the network. In our case, since the network is already completely connected, the one obtained when choosing this function is the same as the initial one (called M-Phase.xml\_cc1).
\subsection{Get strongly connected components}
Plugins$\Rightarrow$BiNoM 2.1$\Rightarrow$BiNoM Analysis$\Rightarrow$Get strongly connected components\\
\begin{figure}
\centering
\includegraphics[width=0.8\textwidth]{graphics/Strongly_Connected_Component_of_M-Phase_network.png}
\caption{Strongly Connected Component of M-Phase network}
\label{Strongly_Connected_Component_of M-Phase_network}
\end{figure}
\\Based on Tarjan’s algorithm\cite{tarjan1972depth}, the strongly connected components are isolated. In simple words, the obtained network, M-Phase.xml\_scc1(figure~\ref{Strongly_Connected_Component_of M-Phase_network}), insures that there exists a path from one node to another and deletes the components which do not respond to this requirement.

\subsection{Prune Graph}
\textbf{Plugins$\Rightarrow$BiNoM 2.1$\Rightarrow$BiNoM Analysis$\Rightarrow$Prune graph}\\
Pruning the graph is equivalent to separating the network into three parts(figure~\ref{Prune_the_graph}: what comes in (M-Phase.xml\_in), what goes out (M-Phase.xml\_out) and the central cyclic part (M-Phase.xml\_scc).
\\This decomposition corresponds to the idea of the bow-tie structure developed by Broder and colleagues\cite{broder2000graph}. In our example, the central cyclic part is the same as figure~\ref{Strongly_Connected_Component_of M-Phase_network}, the strongly connected component. In other cases, it can be composed from several strongly connected components, connected or disconnected.\\
The Prune graph operation decomposes the current network into three parts: IN, OUT and SCC (the later can contain several strongly connected components).
\begin{figure}
\centering
\includegraphics[width=0.8\textwidth]{graphics/Prune_the_graph}
\caption{Prune the graph. (a) Incoming flux: molecules involved in the IN part of the network, and (b) Outgoing flux: molecules involved in the OUT part of the network.}
\label{Prune_the_graph}
\end{figure}

\subsection{Get Material Components}
\textbf{Plugins$\Rightarrow$BiNoM 2.1$\Rightarrow$BiNoM Analysis$\Rightarrow$Get material components}\\
This function uses node name semantics to isolate sub-networks in which each protein takes part. In our example(figure~\ref{Material_Components}), seven sub-networks are created: M-Phase.xml\_Cdc13, M-Phase.xml\_Cdc2, M-Phase.xml\_Rum1, M-Phase.xml\_APC, M-Phase.xml\_Slp1, M-Phase.xml\_Cdc25 and M-Phase.xml\_Wee1. Some major overlaps between sub-networks are expected, as it is the case for Cdc2 and Cdc13 which form a complex.\\
\begin{figure}
\centering
\includegraphics[width=0.8\textwidth]{graphics/Material_components}
\caption{Material Components}
\label{Material_Components}
\end{figure}

\subsection{Get Cycle Decomposition}
\textbf{Plugins$\Rightarrow$BiNoM 2.1$\Rightarrow$BiNoM Analysis$\Rightarrow$Get cycle decomposition}\\
This command decomposes the network into relevant directed cycles\cite{gleiss2001relevant}, using a modification of the Vismara’s algorithm\cite{vismara1997union}. Often, this feature gives information about the life cycle of a protein or a complex, about the feedbacks of the studied network, etc(figure~\ref{Minimal_cycle_decomposition_of_the M-Phase}). Note that the union of all the cycles corresponds to the strongly connected component figure~\ref{Strongly_Connected_Component_of M-Phase_network}.\\
\includegraphics[width=12pt,height=12pt]{graphics/warning} This operation can produce enormous number of cycles! Therefore it is rather suitable for analysis of small to moderate size networks. For a big network, one can start to understand the cyclic network structure by eliminating first the network hubs, which are contained in many network cycles. After that, the local, relatively short, cycles can be represented as meta-nodes (modules) and the analysis for cycles can be repeated.\\
\begin{figure}
\centering
\includegraphics[width=0.8\textwidth]{graphics/Minimal_cycle_decomposition_of_the_M-Phase}
\caption{Minimal cycle decomposition of the M-Phase network.  Cycle 1 includes CDC2 and CDC13 proteins, Cycle 2 CDC25 and Cycle 3 shows the feedback existing between CDC13/CDC2 and CDC25.}
\label{Minimal_cycle_decomposition_of_the M-Phase}
\end{figure}

\subsection{Path Analysis}\label{Path_Analysis}
\textbf{Plugins$\Rightarrow$BiNoM 2.1$\Rightarrow$BiNoM Analysis$\Rightarrow$Path analysis}\\
In a network, it can become handy to find out if there exists a path (or paths) from one species to another, or to verify that a protein or a protein complex is reachable from a starting molecule(figure~\ref{Path_Analysis_All_the_paths}). Provided (an) initial source and target protein(s) that are selected first on the graph then in the dialog window, the command Path analysis can find: the shortest paths, the optimal and suboptimal shortest paths, or all the non-intersecting paths (does not include inner loops), using a finite number of intermediary nodes (use finite breadth search radius), for either directed or undirected paths (figure~\ref{Path_Analysis_Pop-up_window}).\\
\includegraphics[width=12pt,height=12pt]{graphics/warning} In big networks the number of paths can be exponential! It is recommended to find the shortest path first, take its length and increment gradually the breadth search radius starting from this value to find the second shortest, third shortest, etc., paths.\\
\begin{figure}
\centering
\includegraphics[width=0.8\textwidth]{graphics/Path_Analysis_Pop-up_window}
\caption{BiNoM Path Analysis: Pop-up window in which the source(s) and the target(s) need to be specified along with the type of paths (shortest, optimal shortest or all paths).}
\label{Path_Analysis_Pop-up_window}
\end{figure}
\begin{figure}
\centering
\includegraphics[width=0.8\textwidth]{graphics/Path_Analysis_All_the_paths}
\caption{Path Analysis: All the paths leading from one molecular species (Cdc13) to another (Cdc13\_ubi, ubiquitinated form of Cdc13) are highlighted in yellow.}
\label{Path_Analysis_All_the_paths}
\end{figure}

\subsection{Extract subnetwork}
\textbf{Plugins$\Rightarrow$BiNoM 2.1$\Rightarrow$BiNoM Analysis$\Rightarrow$Extract subnetwork}\\
Extract a subnetwork from selected nodes of a network with various options (figure~\ref{Extract_subnetwork_dialog}).
\begin{figure}
\centering
\includegraphics[width=0.8\textwidth]{graphics/Extract_subnetwork_dialog}
\caption{Dialog of extract subnetwork showing options.}
\label{Extract_subnetwork_dialog}
\end{figure}

\subsection{Calc centrality, Inbetweenness undirected, Inbetweenness directed}
\textbf{Plugins$\Rightarrow$BiNoM 2.1$\Rightarrow$BiNoM Analysis$\Rightarrow$Calc centrality$\Rightarrow$Inbetweenness undirected}
\textbf{Plugins$\Rightarrow$BiNoM 2.1$\Rightarrow$BiNoM Analysis$\Rightarrow$Calc centrality$\Rightarrow$Inbetweenness directed}
Display centrality of nodes in cases undirected and directed.

%\subsection{Generate Modular View}
%\textbf{Plugins$\Rightarrow$BiNoM$\Rightarrow$analysis$\Rightarrow$Generate modular view}\\
%Given the initial diagram and some modules (which could be sub-networks of the initial network), it is possible to reconstruct a modular view of the network. For our example, we choose the initial network to be M-Phase.xml and the subparts or modules, the seven sub-networks corresponding to the material components described in (4). From these seven sub-networks only six are selected since two of them, Slp1 and APC, are exactly the same.\\
%The sub-networks or modules need to be specified in the “creating modular view” window (figure~\ref{Modular_view_Pop-up_window}).\\
%\begin{figure}
%\centering
%\includegraphics[width=0.8\textwidth]{graphics/Modular_view_Pop-up_window}
%\caption{BiNoM modular view of the newtork: Pop-up window in which the initial graph and the modules are specified. }
%\label{Modular_view_Pop-up_window}
%\end{figure}
%\\There are different types of modular views. The modules are connected by: (1) the number of shared interactions (figure~\ref{Modular_view_The_resulting_modular_network}, upper panel); (2) the number of shared nodes (reactions + species) for which case the box “Compact module intersection” must be checked (figure~\ref{Modular_view_The_resulting_modular_network}, middle panel); and (3) the shared nodes and reactions showed explicitly (figure~\ref{Modular_view_The_resulting_modular_network}, lower panel).
%\begin{figure}
%\centering
%\includegraphics[width=0.8\textwidth]{graphics/Modular_view_The_resulting_modular_network}
%\caption{BiNoM modular view of the newtork: The resulting modular network (upper panel) with compact module intersections (middle panel) and with explicit intersections (lower panel).}
%\label{Modular_view_The_resulting_modular_network}
%\end{figure}

\subsection{Cluster Networks}
\textbf{Plugins$\Rightarrow$BiNoM 2.1$\Rightarrow$BiNoM Analysis$\Rightarrow$Cluster networks}\\
This command lumps together the modules that share a certain proportion of nodes. At a first glance, it can easily be concluded from Figure~\ref{Modular_view_The_resulting_modular_network} (middle panel) that, for example, the modules M-Phase.xml\_Cdc13 and M-Phase.xml\_Cdc2 share a lot of proteins or protein complexes. Therefore, we can assume that these two modules will collapse into one big module. To determine the clusters, the intersection threshold can be set (from 0 to 100\% intersecting components). For a 30\% intersection threshold, Figure~\ref{Clusters_of_modules_using_the_material_decomposition} is obtained. Four clusters of modules were proposed and linked.\\\\
An alternative modular view has been obtained using the cycle decomposition instead of the material decomposition. The cycles are presented in Figure~\ref{Minimal_cycle_decomposition_of_the M-Phase}. They are obtained by clustering the three cycles into two (cycle 1 + cycle2/cycle3) and organized into a modular view (Figure~\ref{Clusters_of_modules_using_the_cycle_decomposition}).\\
\begin{figure}
\centering
\includegraphics[width=0.8\textwidth]{graphics/Clusters_of_modules_using_the_material_decomposition}
\caption{Clusters of modules. The obtained diagram is a compact modular view of the M-Phase network using the material decomposition and material components clustering}
\label{Clusters_of_modules_using_the_material_decomposition}
\end{figure}
\begin{figure}
\centering
\includegraphics[width=0.8\textwidth]{graphics/Clusters_of_modules_using_the_cycle_decomposition}
\caption{Clusters of modules. The obtained diagram is a compact modular view of the M-Phase network using the relevant cycle decomposition and cycle clustering}
\label{Clusters_of_modules_using_the_cycle_decomposition}
\end{figure}

\subsection{Mono-molecular react.to edges}
\textbf{Plugins$\Rightarrow$BiNoM 2.1$\Rightarrow$BiNoM Analysis$\Rightarrow$Mono-molecular react. to edges}\\
This command transforms monomolecular (with one reactant and one product) reaction nodes into ‘influence’ edges. Thus, monomolecular (linear) reactions are represented as edges and the reaction graph is not bi-partite anymore. When the reaction nodes have the type of influence specified (through the ‘EFFECT’ attribute), the graph is transformed automatically into an influence graph (see Figure~\ref{Network_to_influence_graph}: upper panel: BioPAX network, lower panel, the equivalent influence network). Non-linear non-monomolecular reactions (such as complex assemblies) are not transformed and remain to be represented as network nodes.
\begin{figure}
\centering
\includegraphics[width=0.8\textwidth]{graphics/Network_to_influence_graph}
\caption{From a BioPAX network (upper panel) to an influence graph (lower panel).}
\label{Network_to_influence_graph}
\end{figure}

\subsection{‘Linearize’ network}
\textbf{Plugins$\Rightarrow$BiNoM 2.1$\Rightarrow$BiNoM Analysis$\Rightarrow$‘Linearize’ network}\\
Remove reactions and reconnect edges according to a supposed influence (figure~\ref{Linearized_Network_M-Phase})\\
\includegraphics[width=12pt,height=12pt]{graphics/warning} The got network is not an influence network in the biological sense. But, it can be used to build an influence network.
\begin{figure}
\centering
\includegraphics[width=0.8\textwidth]{graphics/Linearized_Network_M-Phase}
\caption{Result of applying "Linearize network" to M-Phase.}
\label{Linearized_Network_M-Phase}
\end{figure}

\subsection{Exclude intermediate nodes}
\textbf{Plugins$\Rightarrow$BiNoM 2.1$\Rightarrow$BiNoM Analysis$\Rightarrow$Exclude intermediate nodes}\\
This function opens a dialog where nodes to be excuded can be selected (figure~\ref{Exclude_nodes_Dialog}). It creates a network without the selected nodes and reconnects edges.
\begin{figure}
\centering
\includegraphics[width=7 cm]{graphics/Exclude_nodes_Dialog}
\caption{Dialog to select nodes to be excluded in the created network.}
\label{Exclude_nodes_Dialog}
\end{figure}

\subsection{Extract Reaction Network}
\textbf{Plugins$\Rightarrow$BiNoM 2.1$\Rightarrow$BiNoM Analysis$\Rightarrow$Extract reaction networks}\\
This function cleans up the diagram to only keep the reaction network. Only nodes with ‘XXXX\_REACTION’ and ‘XXXX\_SPECIES’ attributes (where XXXX stands for any word) are kept as a result of this operation. For example, it helps to clean the reaction network interface from the result of querying BioPAX index (which contains many other node types such as entities and publications.

\subsection{Path consistency analysis}
\textbf{Plugins$\Rightarrow$BiNoM 2.1$\Rightarrow$BiNoM Analysis$\Rightarrow$Path consistency analysis}\\
This function is based on an algorithm of  Path Influence Quantification (PIQuant). Shortly, for every annotated nodes, the influence is computed by summing (optimal and sub-optimal) paths contribution (1/(length of the path)), multiplied by the nodes annotation that represent biological data. For global influence computation, influences are summed on all over the annotated nodes. The figure \ref{PIQuant_example} shows the computing in a simple example.\\\\
\includegraphics[width=12pt,height=12pt]{graphics/warning} The edge attribute which signs influence is "EFFECT" which can take 2 values EFFECT:activation and EFFECT:inhibition. An AIN file updates this attribute (section~\ref{Import_AIN_file}) or it can be imported as an edge attribute by Cytoscape.\\\\
\begin{figure}
\centering
\includegraphics[width=0.8\textwidth]{graphics/PIQuant_example}
\caption{Computing influence in a simple example.}
\label{PIQuant_example}
\end{figure}
For global influences, a p-value can be computed, by shuffling the annotation. Among the possible options accessible in BiNoM, the possibility of removing inconsistent path (paths that have an intermediate node with inconsistent annotation) was sometimes used.\\\\
The dialog has 2 steps: input parameters and options, select path and display results.\\\\
Step 1 see figure~\ref{Path_consistency_analyser_Dialog1}:
\begin{itemize}
\item Select activity attribute name and update list, only active nodes are displayed in box.
\item Targets are phenotypes, any node can be selected.
\item Choose options of paths, see glossary in section \ref{GLOSSARY}.
\item  Click OK.
\end{itemize}
\begin{figure}
\centering
\includegraphics[width=0.8\textwidth]{graphics/Path_consistency_analyser_Dialog1}
\caption{Path consistency analysis: dialog of step 1, select attribute, sources, targets and options.}
\label{Path_consistency_analyser_Dialog1}
\end{figure}
Step 2 see figure~\ref{Path_consistency_analyser_Dialog2}:
\begin{itemize}
\item Click on a target node, paths and features of path evaluated according to PIQuant are displayed.
\item A text report about optimal cut set can be get by "path activities".
\item Possibility to filter nodes and paths.
\item A  significance test by  p-value allows to insure got activities are significant.
\end{itemize}
\begin{figure}
\centering
\includegraphics[width=0.8\textwidth]{graphics/Path_consistency_analyser_Dialog2}
\caption{Path consistency analysis: dialog of step 2, display paths and activities, get all results}
\label{Path_consistency_analyser_Dialog2}
\end{figure}  

\subsection{OCSANA analysis}
\textbf{Plugins$\Rightarrow$BiNoM 2.1$\Rightarrow$BiNoM Analysis$\Rightarrow$OCSANA analysis}\\

Work in progress.

\subsection{Create neighborhood sets file}
\textbf{Plugins$\Rightarrow$BiNoM 2.1$\Rightarrow$BiNoM Analysis$\Rightarrow$Create neighborhood sets file}\\
This function creates a file *.gmt (a text file where nodes are separated by \textless Tab\textgreater) containing the neighbors of selected nodes according to option of the dialog(figure~\ref{Create_Neigborhood_File_Dialog})
\begin{figure}
\centering
\includegraphics[width=7 cm]{graphics/Create_Neigborhood_File_Dialog}
\caption{Diialog for options of creating a neighborhood sets file.}
\label{Create_Neigborhood_File_Dialog}
\end{figure}  
\clearpage

\section{BiNoM Fading Signal Propagation Model}
This menu provides functions of computing and visualizing influence according to a simple model describe in ~\ref{Influence_fade_model}). Before using these functions some parameters must be input as weight of edges and influence reach.

\subsection{Select Sub-network from Sources to Targets}
\textbf{Plugins$\Rightarrow$BiNoM 2.1$\Rightarrow$BiNoM fade Influence$\Rightarrow$Select Sub-Network From Sources to Targets}\\
This function selects nodes and edges in the network between a list of sources and a list of targets.  Loops are included in the selection. For all nodes as sources or targets: select the first node and type shift+control+end.

\subsection{Update Influence Attribute}
\textbf{Plugins$\Rightarrow$BiNoM 2.1$\Rightarrow$BiNoM fade Influence$\Rightarrow$Update Weigth Influence Attribute}\\
A weight of influence as edge attribute must be affected to every edge. This dialog updates the weight attribute by selecting an  attribute et affecting their values to activation (weight=+1) and inhibition (weight=-1), 3 possible values for attribute. Generally, the attribute is "interaction" and the values "activation" or "inhibition". Be careful of lower-case and upper-case (see~\ref{Weight_Attribute_Dialog}).

\begin{figure}
\centering
\includegraphics[width=0.7\textwidth]{graphics/Weight_Attribute_Dialog}
\caption{Dialog updating weight attribute from an other attribute}
\label{Weight_Attribute_Dialog}
\end{figure}

\subsection{Input Reach Parameter}
\textbf{Plugins$\Rightarrow$BiNoM 2.1$\Rightarrow$BiNoM fade Influence$\Rightarrow$Input Reach Parameter}\\
Input the number of paths beyond which the influence is insignificant, less than 5\%. It a real number.

\subsection{Display Network and Parameter Features}
\textbf{Plugins$\Rightarrow$BiNoM 2.1$\Rightarrow$BiNoM fade Influence$\Rightarrow$Display Network and Parameter Features}\\
Display in a text box: size of network, reach parameter, min, max and mean influence. Useful to insure the realistic value of reach.

\subsection{Display Influence Array As Text}
\textbf{Plugins$\Rightarrow$BiNoM 2.1$\Rightarrow$BiNoM fade Influence$\Rightarrow$Display Influence Array As Text}\\
The influence matrix is displayed in a text box which can be copied in the clipboard and paste in a spreadsheet, sources in columns, targets in rows, names in alphabetical order. Parameters are in title.\\\\
Text window with 2 options: 
\begin{itemize}
\item for visualizing, "nc" = not connected, only 3 digits after point for numbers,
\item for computing,  all values are numeric with all possible digits,
\end{itemize}
Same dialog as "Select Sub-network from Sources to Targets". Furthermore, structural sources and targets are preselected (respectively in degree=0, out degree=0).

\subsection{Display Influence Array as Paved Window}
\textbf{Plugins$\Rightarrow$BiNoM 2.1$\Rightarrow$BiNoM fade Influence$\Rightarrow$Display Influence Array as Paved Window}\\
Same computing as "Display Influence Array As Text". Results are in:
\begin{itemize}
\item paved window with 2 options: 
\begin{itemize}
\item activated in red, inhibited in green, light to dark according to the value, not connected in black,
\item activated in red, inhibited in blue, light to dark according to the value, not connected in white (see~\ref{paved_window}),
\end{itemize}
\item text window where are displayed details of selected area in paved window.
\end{itemize}

\begin{figure}
\centering
\includegraphics[width=1.0\textwidth]{graphics/paved_window}
\caption{Window paved by the level of influence between species}
\label{paved_window}
\end{figure}

\subsection{Influence by Active Nodes as Attribute}
\textbf{Plugins$\Rightarrow$BiNoM 2.1$\Rightarrow$BiNoM fade Influence$\Rightarrow$Influence by Active Nodes as Attribute}\\
Activity levels of nodes are input in an attribute "ACTIV\_IN". The result of the multiplication of influence matrix by activity level input is "ACTIV\_OUT" attribute.

\subsection{Display Influence Reach Area in Array}
\textbf{Plugins$\Rightarrow$BiNoM 2.1$\Rightarrow$BiNoM fade Influence$\Rightarrow$Display Influence Reach Area in Array}\\
Computing as "Display Influence Array for Computing" with all weights=1. Useful to appreciate the absolute level of influence by a species to other species.

\subsection{Influence Reach Area as Attribute}
\textbf{Plugins$\Rightarrow$BiNoM 2.1$\Rightarrow$BiNoM fade Influence$\Rightarrow$Influence Reach Area as Attribut}\\
Computing from selected nodes with all weights=1. Absolute influence levels are put in attribute "INFLUENCE\_AREA\_N" where N keeps every successive results. Start nodes must be noted manually. Useful to visualize the influence of a group of species.

\clearpage

\section{BiNoM Module Manager}
Module manager is useful for creating modular view of large networks without loosing details of modules (using “nest”, object of Cytoscape v7 and after).

\subsection{Create Network of Modules}
\textbf{Plugins$\Rightarrow$BiNoM 2.1$\Rightarrow$BiNoM Module Manager$\Rightarrow$Create Network of Modules}\\
Create a new network from a list of sub-networks (sub-networks are selected in the network list see figure~\ref{Create_network_of_modules}).\\
Nodes=modules, no edge. Visual style created in VizMapper for module network . The got network is as \ref{M-Phase_Material_Modular} without edge and with nodes on grid.\\
\includegraphics[width=20pt,height=20pt]{graphics/warning} Module names and node names must be different, all network names too.\\\\
To go from module to sub-network:\\
select node$\Rightarrow$CRight click$\Rightarrow$CNested Network$\Rightarrow$Go to Nested Network.
\begin{figure}
\centering
\includegraphics[width=0.8\textwidth]{graphics/Create_network_of_modules}
\caption{Dialog for select networks to become modules in modular0}
\label{Create_network_of_modules}
\end{figure}

\subsection{Create Connections between Modules}
\textbf{Plugins$\Rightarrow$BiNoM 2.1$\Rightarrow$BiNoM Module Manager$\Rightarrow$Create Connections between Modules}\\
Create edges linking modules from all edges of the selected network.\\
Links are simplified, no distinction between left and right (molecule flow), no duplication if same interaction.\\
Warning message if duplicated or absent nodes (may disturb links).
\begin{figure}
\centering
\includegraphics[width=0.8\textwidth]{graphics/M-Phase_Material_Modular}
\caption{M-Phase is divided into modules by get material component. The modular view is got by creating network of modules with organic layout. The function "Create connections between modules" links modules according to the reference network. The function "Find common nodes in modules" creates intersection edges. }
\label{M-Phase_Material_Modular}
\end{figure}

\subsection{Create Modules from Networks}
\textbf{Plugins$\Rightarrow$BiNoM 2.1$\Rightarrow$BiNoM Module Manager$\Rightarrow$Create Modules from Networks}\\
Create modules in the active network from a list of sub-networks (sub-networks are selected in the network list)\\
All edges are kept. See edge attribute PREVIOUS\_ID for their origin.\\
The attribute BIOPAX\_NODE\_TYPE is set to “pathway” (see visual style BiNoM BioPAX).\\
\includegraphics[width=20pt,height=20pt]{graphics/warning} All nodes of sub-networks must be found once in the active network (no intersection between sub-networks).

\subsection{Agglomerate the Nearest Nodes in Modules}
\textbf{Plugins$\Rightarrow$BiNoM 2.1$\Rightarrow$BiNoM Module Manager$\Rightarrow$Agglomerate the Nearest Nodes in Modules}\\
Create modules and a modular view by agglomerating the nearest nodes in the active network (see~algorithm~in section~\ref{Agglomeration_by_shortest_path}.\\\\
Input 2 parameters to get not too big sub-networks containing not too far nodes:
\begin{itemize}
\item Maximal distance between nodes or modules in number of edges,
\item Maximal number of nodes in modules.
\end{itemize}
Confirm creation if agree with displayed result (see dialog~\ref{Agglomerate_in_modules_dialog}).\\\\
Sub-networks are created and gathered in a packed network as the function "Create modules from networks" (see figure~\ref{M-Phase_packed}).
\begin{figure}
\centering
\includegraphics[width=0.8\textwidth]{graphics/Agglomerate_in_modules_dialog}
\caption{This window displays modules, number of node, last distance and number of links of agglomerating process. Yes lauches the process of agglomerating.}
\label{Agglomerate_in_modules_dialog}
\end{figure}
\begin{figure}
\centering
\includegraphics[width=0.8\textwidth]{graphics/M-Phase_packed}
\caption{M-Phase is got by creating network from modules, modules created by agglomerating the nearest nodes (maximal distance=1, maximal size=12 nodes).}
\label{M-Phase_packed}
\end{figure}

\subsection{List Nodes of Modules and Network}
\textbf{Plugins$\Rightarrow$BiNoM 2.1$\Rightarrow$BiNoM Module Manager$\Rightarrow$List Nodes of Modules and Network}\\
List nodes of network and nodes included in modules.\\
Result in text box can be simply copied in a spreadsheet through clipboard.

\subsection{Find Common Nodes in Modules}
\textbf{Plugins$\Rightarrow$BiNoM 2.1$\Rightarrow$BiNoM Module Manager$\Rightarrow$Find Common Nodes in Modules}\\
Display in text box the belonging matrix of nodes (modules in columns, nodes in rows, size of modules in last row, frequency in modules in last column); result more easily usable after copying in a spreadsheet (see~\ref{Common_nodes_in_modules}.\\\\
Create intersection edges with number of common nodes as attribute (COMMON\_NODES), green edges in figure~\ref{M-Phase_Material_Modular}.\\\\
Create node attribute containing the node numbers of modules (NODE\_NUMBER).\\\\
Module Visual StyleCan be adapted to the wished visual aspect by hands in VizMapper, for example:
\begin{itemize}
\item To visualize NODE\_NUMBER: double click Node Size, select NODE\_NUMBER, continuous mapping, adjust width by graphical view.
\item To visualize COMMON\_NODES double click Edge Line Width, select COMMON\_NODES, continuous mapping, adjust width by graphical view.
\end{itemize}
\begin{figure}
\centering
\includegraphics[width=0.8\textwidth]{graphics/Common_nodes_in_modules}
\caption{Matrix of nodes: modules in columns, nodes in rows, size of modules in last row, frequency in modules in last column.}
\label{Common_nodes_in_modules}
\end{figure}

\subsection{Assign Module Names to Node Attribute}
\textbf{Plugins$\Rightarrow$BiNoM 2.1$\Rightarrow$BiNoM Module Manager$\Rightarrow$Assign Module Names to Node Attribute}\\
Create a node attribute (named as the modular network), containing module names. This attribute may be used to visualize modules in the reference network.

\subsection{List Components of Species in Network and Modules}
\textbf{Plugins$\Rightarrow$BiNoM 2.1$\Rightarrow$BiNoM Module Manager$\Rightarrow$List Components of Species in Network and Modules}\\
List components of species (their names must respect BiNoM syntax). Useful to name modules.

\subsection{Create Network from Union of Selected Modules}
\textbf{Plugins$\Rightarrow$BiNoM 2.1$\Rightarrow$BiNoM Module Manager$\Rightarrow$Create Network from Union of Selected Modules}\\
Create a network from union of selected modules and its corresponding module in the current network (named by module names separated by \&).

\subsection{Create Network from Intersection of 2 Selected Modules}
\textbf{Plugins$\Rightarrow$BiNoM 2.1$\Rightarrow$BiNoM Module Manager$\Rightarrow$Create Network from Intersection of 2 Selected Modules}\\
Create a network from intersection of 2 selected modules and its corresponding module (named by module names separated by \textbar.\\\\
Confirm for deleting the common nodes in the selected modules.

\subsection{Recreate Lost Connections inside Modules}
\textbf{Plugins$\Rightarrow$BiNoM 2.1$\Rightarrow$BiNoM Module Manager$\Rightarrow$Recreate Lost Connections Inside Modules}\\
Recreate connections inside modules which may have been lost by modularizing operations.

\subsection{Destroy Networks Unused as Module}
\textbf{Plugins$\Rightarrow$BiNoM 2.1$\Rightarrow$BiNoM module manager$\Rightarrow$Destroy Networks Unused as Module}\\
Select networks to be deleted among a list of networks which are not used as modules in the current network (simplify cleaning session).

\clearpage

\section{BiNoM BioPAX3 Utils}
\subsection{BioPAX 3 Property Editor} \label{BioPAX_Property_Editor}
\textbf{Plugins$\Rightarrow$BiNoM 2.1$\Rightarrow$BiNoM BioPAX 3 Utils$\Rightarrow$ BioPAX 3 Property Editor}\\

All the information available in a BioPAX file can be easily retrieved using the
BioPAX Property Editor function. A component on the diagram must be selected
first (CDC2 in Figure \ref{BioPAX_Property_Editor_cdc2}) and a window appears
with all available information concerning the molecule.\\\\

\begin{figure}[h]
\centering
\includegraphics[width=0.8\textwidth]{graphics/BioPAX_Property_Editor_cdc2}
\caption{BioPAX Property Editor: example of the properties concerning CDC2
component in M-Phase model}
\label{BioPAX_Property_Editor_cdc2}
\end{figure}


\parbox{\textwidth}{ In the menu of the Property Editor, several options are
offered:

\begin{itemize}
\item Display valid attributes / Display all attributes hides all the empty
fields (for example, in Figure \ref{BioPAX_Property_Editor_apoptosis}:
Availability or Evidence have \textless empty object list\textgreater and would
be hidden) / shows all the available fields, even the empty ones.
\item  \textless\textless~and \textgreater\textgreater~correspond to back or
forward buttons and follow the historical exploration of the Property Editor
(similar to ‘Back’ and ‘Forward’ buttons of a network browser).
\item Close current tab or Close all tabs closes the current page of the
property editor or all the open pages.
\item  Display / Edit shows a simple display of the page editor where no change
can be made (Figure \ref{BioPAX_Property_Editor_apoptosis}) / allows changes in
the fields by adding, removing or updating information (Figure
\ref{BioPAX_Property_Editor_cdc2}). For the latter, click first on the Edit tab
on the upper menu, then on update situated near the field to modify. In Figure
\ref{BioPAX_Property_Editor_cdc2}, as an example, some comments were added
manually: “CDC2 is a kinase that binds to CDC13 to form a dimer”. In the
Apoptosis example (Figure \ref{BioPAX_Property_Editor_apoptosis}), extensive
information is already available concerning the pathway, references, etc.
\end{itemize}}
\begin{figure}[h]
\centering
\includegraphics[width=0.8\textwidth]{graphics/BioPAX_Property_Editor_apoptosis}
\caption{BioPAX Property Editor: example of “apoptosis pathway” node properties}
\label{BioPAX_Property_Editor_apoptosis}
\end{figure}	
For more details on BioPAX description standard, visit the webpage:
http://www.biopax.org/ 
\subsection{BioPAX 3 Class Tree}
\textbf{Plugins$\Rightarrow$BiNoM 2.1$\Rightarrow$BiNoM BioPAX 3 Utils$\Rightarrow$BioPAX 3 Class
Tree}\\
All the statistics concerning the pathway are listed: the number of reactions,
associations or catalyses, the number of proteins or complexes, etc
(figure~\ref{BioPAX_Class_Tree}). More information can be accessed by selecting
a specific object which, when clicked on, leads to the BioPAX 3 Property Editor
window (see section~\ref{BioPAX_Property_Editor}).\\\\
\begin{figure}[h]
\centering
\includegraphics[width=0.8\textwidth]{graphics/BioPAX_Class_Tree}
\caption{BioPAX Class tree. On the left frame, the model is described in terms
of interactions, entities, etc. On the right frame, the proteins, selected in
the left frame, are listed. The links are clickable and open a BioPAX Property
Editor window.}
\label{BioPAX_Class_Tree}
\end{figure}
To complete the network, the user can easily add new information or a new
protein, protein complex, type of interaction, etc., by clicking on the New
Instance tab.

\subsection{Use Simplified URI Names}
\textbf{Plugins$\Rightarrow$BiNoM 2.1$\Rightarrow$BiNoM BioPAX 3 Utils$\Rightarrow$Use Simplified URI Names}\\

In the BioPAX Class Tree, protein names can have either URI names (Uniform Resource Identifier used to give a unique identification to proteins) or “BiNoM Naming Service” names. For example, for the apoptosis pathway, the protein BAD is referred to as\\

“\textit{UniProt\_Q92934\_Bcl2\_antagonist\_of\_cell\_death\_BAD\_\_Bcl\_2\_binding\_component\_6\_\_Bcl\_\_XL\_Bcl\_2\\
\_associated\_death\_promoter\_\_Bcl\_2\_like\_8\_protein}”\\

in the URI case and just “BAD” in the BiNoM Naming Service case. For the rules of how BiNoM generates names see section~\ref{BiNoM_Naming_Service}.

\subsection{Synchronize networks with BioPAX 3}
\textbf{Plugins$\Rightarrow$BiNoM 2.1$\Rightarrow$BiNoM BioPAX 3 Utils$\Rightarrow$Synchronize networks with BioPAX 3}\\
This command updates all the interfaces according to the changes made to individual BioPAX objects.
\clearpage

\section{BiNoM BioPAX3 Query}
The purpose of the functions related to the query language is to work with huge BioPAX files and extract from the BioPAX documents only the information that is of interest. For this part, we will use the apoptosis example initially extracted from Reactome database: Apoptosis.owl. This set of functions can be used with big pathway databases already exported to BioPAX: Reactome, BioCyc, NetPath (see http://www.biopax.org for the complete list).

\subsection{Generate Index}
\textbf{Plugins$\Rightarrow$BiNoM BioPAX 3 Query$\Rightarrow$Generate Index}\\
Using this function BiNoM maps the content of BioPAX file onto a labeled graph (referred to as index). It creates an *.xgmml file from an *.owl one (figure~\ref{Generate_BioPAX_Index}). For the definition of BioPAX index, see section~\ref{Standard_BioPAX_Interfaces}.
\begin{figure}[h]
\centering
\includegraphics[width=18 cm]{graphics/Generate_BioPAX_Index}
\caption{Generate BioPAX Index.}
\label{Generate_BioPAX_Index}
\end{figure}

\subsection{Load Index}
\textbf{Plugins$\Rightarrow$BiNoM BioPAX 3 Query$\Rightarrow$Load Index}\\
Once the xgmml is created, it can be loaded into memory. The index is global object, i.e. only one index can be used at a time.Load Index loads the index file from xgmml format (figure~\ref{Load_Index_Dialog}).\\\\
\begin{figure}[h]
\centering
\includegraphics[width=18 cm]{graphics/Load_Index_Dialog}
\caption{Load Index dialog.}
\label{Load_Index_Dialog}
\end{figure}
Together with the index, you can also upload a tab-delimited “accession number file” which corresponds to a list of synonyms for the genes/proteins ids used in a network (see an example of the content of some accession number file at figure~\ref{Accession_Number_File}). An entity in the index can be identified by its id, by any XREF attribute (see section~\ref{Standard_BioPAX_Interfaces}), by node name, or by any synonym from the accession table (if it is provided).
\begin{figure}[h]
\centering
\includegraphics[width=8 cm]{graphics/Accession_Number_File}
\caption{Example of accession Number file. First column is a synonym (which can have structure $\textless database\textgreater :\textless standard_id\textgreater)$, the second column is the id used inside the BioPAX file.}
\label{Accession_Number_File}
\end{figure}

\subsection{Display Index Info}
\textbf{Plugins$\Rightarrow$BiNoM BioPAX 3 Query$\Rightarrow$Display Index Info}\\
This command opens a window indicating the name of the graph, the name of the file, the accession number file, when available, the number of records, and the various statistics of the index: number of publications, proteins, physicalEntities, complexes, biochemical reactions, pathways, pathwaySteps, catalyses, and modulations (necessary proteins for catalyses). See figure~\ref{BioPAX_Index_Info}.
\begin{figure}[h]
\centering
\includegraphics[width=18 cm]{graphics/BioPAX_Index_Info}
\caption{Display Index info.}
\label{BioPAX_Index_Info}
\end{figure}

\subsection{Select Entities}
\textbf{Plugins$\Rightarrow$BiNoM BioPAX 3 Query$\Rightarrow$Select Entities}\\
The BioPAX document is often too big to find the protein or gene that needs to be studied. To access it easily and rapidly, it is possible to find the component directly with this command and build a specific network around that molecule.\\\\
For example, in Apoptosis.xgmml, we choose to find the caspases 8 and extend the network around it. When choosing Plugins => BiNoM BioPAX Query => Select entities from the index, a dialog window pops up and offers the possibility to find a protein or a gene by its name or id or XREF attribute or synonym, from the current network when a network is already opened, or from the list of identities associated with the BioPAX index (figure~\ref{Select_entities_from_index}).\\\\
For our example, we choose the second option. To increase the probability to find the protein in the list, we propose, in figure~\ref{Select_entities_from_index}, three different versions of the same name: CASP8, Caspase8 or caspases\_8, all separated by space (the separator can be also comma and semi-colon or line-break symbol). One of them (CASP8) corresponds to the name from the BioPAX list and a new network is created with only one protein, CASP8 (= MCH5 in the index), at the center of it. The other ones were not found (see output in figure~\ref{xxx}). It is also possible to select more than one entity, in this case, the components all appear in the same window.\\\\
The output is chosen to appear in a new network (selection is made at the bottom of the dialog window in figure~\ref{Select_entities_from_index}) but it is also an option to view several genes or proteins in the same network by checking “output in the current network”.\\\\
\begin{figure}[h]
\centering
\includegraphics[width=14 cm]{graphics/Select_entities_from_index}
\caption{Select entities from the index.}
\label{Select_entities_from_index}
\end{figure}
A network is created with only one node, caspase 8, called MCH5 in the index. Note that for this part, it is advised to use the BiNoM BioPAX visual style to view the resulting network.

\subsection{Standard Query}
\textbf{Plugins$\Rightarrow$BiNoM BioPAX 3 Query$\Rightarrow$Standard Query}\\
This command proceeds through a series of actions that will extend or make the studied network more specific to the user’s needs.\\\\
Let’s start with diverse queries from the network created for the Caspase 8 entity. A dialog window opens as figure~\ref{Standard_Query_Dialog}.\\\\
\begin{figure}[h]
\centering
\includegraphics[width=12 cm]{graphics/Standard_Query_Dialog}
\caption{BioPAX Standard Query: dialog window.}
\label{Standard_Query_Dialog}
\end{figure}
All the options proposed in the dialog window (figure~\ref{Standard_Query_Dialog}) are listed here:
\subsubsection{In the input section}
All nodes / Selected nodes: In the network, you can submit queries that concern all the nodes in the network or only the selected nodes (highlighted in yellow). 
\subsubsection{Once you decide on which proteins you wish to work, you can:}
\begin{itemize}
\item Add complexes
\begin{itemize}
\item “no expand” adds only the homodimers of the molecule (figure~\ref{Standard_Query_Add_complexes}a). If several proteins were queried, then all hetero-dimers in which all the proteins participate would appear.
\item “expand” adds all the complexes in which MCH5 is involved (figure~\ref{Standard_Query_Add_complexes}b). The green arrow with a diamond ending represents the inclusion of one protein in a complex form.
\end{itemize}
\begin{figure}[h]
\centering
\includegraphics[width=18 cm]{graphics/Standard_Query_Add_complexes}
\caption{BioPAX Standard Query: Add complexes with (a) the « no expand” or (b) “expand” option.}
\label{Standard_Query_Add_complexes}
\end{figure}
\item Add chemical species\\
This function adds, for each species, the cellular location and its specified modifications. It is linked to the protein with a grey edge (see figure~\ref{Standard_Query_Chemical_species}).
\begin{figure}[h]
\centering
\includegraphics[width=18 cm]{graphics/Standard_Query_Chemical_species}
\caption{Chemical species. Cellular locations of all forms of MCH5.}
\label{Standard_Query_Chemical_species}
\end{figure}
\item Add reactions
\begin{itemize}
\item “connecting reactions” connects all present species that have common reactions (connected by a yellow node in figure~\ref{Standard_Query_All_connecting_reactions}).
\begin{figure}[h]
\centering
\includegraphics[width=18 cm]{graphics/Standard_Query_All_connecting_reactions}
\caption{Adding reactions: example when all connecting reactions.}
\label{Standard_Query_All_connecting_reactions}
\end{figure}
\item “all reactions” includes all the reactions involving the chemical species (figure~\ref{Standard_Query_Adding_all_reactions}).
\begin{figure}[h]
\centering
\includegraphics[width=18 cm]{graphics/Standard_Query_Adding_all_reactions}
\caption{Adding reactions: example adding all reactions.}
\label{Standard_Query_Adding_all_reactions}
\end{figure}
\item “make reactions complete” adds all the sources and targets of the reactions (figure~\ref{Standard_Query_Making_reactions_complete}) listed in the BioPAX index, including, for example, the pathway nodes and publications links.
\begin{figure}[h]
\centering
\includegraphics[width=18 cm]{graphics/Standard_Query_Making_reactions_complete}
\caption{Adding reactions: example when making the reactions complete.}
\label{Standard_Query_Making_reactions_complete}
\end{figure}
\end{itemize}
\item Add publications\\
When available, this function adds all the references associated with a reaction (see figure~\ref{Standard_Query_Adding_publications}).
\begin{figure}[h]
\centering
\includegraphics[width=18 cm]{graphics/Standard_Query_Adding_publications}
\caption{Adding publications.}
\label{Standard_Query_Adding_publications}
\end{figure}
\end{itemize}
\subsubsection{Output}
The result of the queries can be seen either in the current network or in a new network.

\parbox{\textwidth}{
\subsection{Index Path Analysis}
\textbf{Plugins$\Rightarrow$BiNoM BioPAX 3 Query$\Rightarrow$Index Path Analysis}\\
This command finds the directed or non-directed, shortest, optimal or suboptimal, non intersecting paths with a pre-defined number of intermediaries in an index file. Note that the species need to be selected on a graph before this query.\\\\
This part of the query engine uses the same algorithms and options as Path analysis dialog (see section~\ref{Path_Analysis}), however, with the network (index) kept completely in memory, without explicit visualization. Moreover, the network is slightly modified before this type of query: in particular, all non-directed edges (of CONTAINS, SPECIESOF and some other types) are represented as bi-directional, some nodes (publications and, optionally, smallMolecules) are removed.\\\\
For example, the following steps
\begin{enumerate}
\item Select Entities: specify CASP8 and FAS proteins.
\item Select the two nodes.
\item Index Path analysis: Find all non-intersecting paths.
\item BiNoM Analysis: Extract Reaction Network.
\end{enumerate}
produce the following network connecting CASP8 and FAS proteins (it also involves FADD because it complexes with FAS and CASP8 on the membrane). See figure~\ref{Index_Path_Analysis}.
}
\begin{figure}[h]
\centering
\includegraphics[width=18 cm]{graphics/Index_Path_Analysis}
\caption{Index path Analysis with CASP8 and FAS.}
\label{Index_Path_Analysis}
\end{figure}

\subsection{View Query Log}
\textbf{Plugins$\Rightarrow$BiNoM BioPAX 3 Query$\Rightarrow$}\\
In this window (figure~\ref{BioPAXViewQueryLogDialog}), are recapitulated all the queries done during the session.
\begin{figure}[h]
\centering
\includegraphics[width=18 cm]{graphics/BioPAXViewQueryLogDialog}
\caption{View Query Log Dialog}
\label{BioPAXViewQueryLogDialog}
\end{figure}
\clearpage

\section{BiNoM Utilities}
There are various functions that facilitate the manipulation of the networks, the copying and pasting of its subparts, the selection of some portions of it, etc. Those of them are missing in Cytoscape.
\subsection{Select Edges between Selected Nodes}
\textbf{Plugins$\Rightarrow$BiNoM 2.0$\Rightarrow$BiNoM Utilities$\Rightarrow$Select Edges between Selected Nodes} or f8\\
When some components are selected by their names (Select$\Rightarrow$Nodes$\Rightarrow$By Name) or simply with the mouse, the edges between the nodes are not selected. This function allows to remedy this problem.\\\\
This might be especially useful when the selection is copied and pasted in another network, although it is possible to paste the nodes with the edges connecting them, without selecting them, by choosing File$\Rightarrow$New$\Rightarrow$Network$\Rightarrow$From selected nodes, all edges when pasting them.

\subsection{Select upstream neighbours}
\textbf{Plugins$\Rightarrow$BiNoM 2.0$\Rightarrow$BiNoM Utilities$\Rightarrow$Select upstream neighbours} or ctrl+8\\
Select upstream neighbours of selected nodes. Whole upstream neighbours can be selected by repeating the command until no more node is selected.

\subsection{Select downstream neighbours}
\textbf{Plugins$\Rightarrow$BiNoM 2.0$\Rightarrow$BiNoM Utilities$\Rightarrow$Select downstream neighbours} or ctrl+9\\
Select downstream neighbours of selected nodes. Ditto downstream.

\subsection{Double Network Differences}
\textbf{Plugins$\Rightarrow$BiNoM 2.0$\Rightarrow$BiNoM Utilities$\Rightarrow$Double Network Differences}\\
A network is composed of nodes and edges. With this command, two networks A and B can be compared and the difference observed between the two is created in two new graphs: A-B for the differences of A compared to B (A-A$\cap$B), and B-A for the differences of B compared to A (B-A$\cap$B). Note that in the output networks, the layout of the first graph is conserved.

\subsection{Update Networks}
\textbf{Plugins$\Rightarrow$BiNoM 2.0$\Rightarrow$BiNoM Utilities$\Rightarrow$Update Networks}\\
When a network is modified, it is possible to update all the networks that are related to it, either because they are modules, sub-networks or older versions of it. That way, any changes, additions or deletions in a network can be propagated to the sub-parts that are derived from the initial version of that network. The user specifies which network is added, which one is deleted (when necessary) and which networks need to be updated in the proposed list.\\\\
Whatever is added or deleted in each sub-network is presented to the user in a separate window before any action is made. The user can agree or disagree by checking or unchecking the box in front of the network names. The previous version of the updated diagrams will not be deleted but saved under \textit{network\_name.xml\_old}. 

\subsection{Update connections from other network}
\textbf{Plugins$\Rightarrow$BiNoM 2.0$\Rightarrow$BiNoM Utilities$\Rightarrow$Update connections from other network}\\
The dialog (figure~\ref{Update_connections}) propose to select:
\begin{itemize}
\item From network, the reference network where are the connections.
\item Networks to update: networks in which connections are copied.
\end{itemize}
\begin{figure}[h]
\centering
\includegraphics{graphics/Update_connections}
\caption{Update connections from other networks Dialog}
\label{Update_connections}
\end{figure}
Some connections may be duplicated. So a Cytoscape command (version 8.0) may be used to delete them:\\
\textbf{Plugins$\Rightarrow$Network Modifications$\Rightarrow$Remove Duplicated Edges}

\subsection{Merge Networks and Filter by Frequency}
\textbf{Plugins$\Rightarrow$BiNoM 2.0$\Rightarrow$BiNoM Utilities$\Rightarrow$Merge Networks and Filter by Frequency}\\
Create a network by merging the selected networks where the percentage of common nodes is geater then intersection threshold (see dialog figure~\ref{Merge_Networks_and_Filter}) .\\\\
\begin{figure}[h]
\centering
\includegraphics{graphics/Merge_Networks_and_Filter}
\caption{Merge Networks and Filter by Frequency Dialog}
\label{Merge_Networks_and_Filter}
\end{figure}

\subsection{Clipboard}
The description of the commands, which work only inside a Cytoscape session, is self-explanatory.\\
\textbf{Plugins$\Rightarrow$BiNoM 2.0$\Rightarrow$BiNoM Utilities$\Rightarrow$Clipboard$\Rightarrow$Copy selected nodes and edges to clipboard}\\
\textbf{Plugins$\Rightarrow$BiNoM 2.0$\Rightarrow$BiNoM Utilities$\Rightarrow$Clipboard$\Rightarrow$Add selected nodes and edges to clipboard}\\
\textbf{Plugins$\Rightarrow$BiNoM 2.0$\Rightarrow$BiNoM Utilities$\Rightarrow$Clipboard$\Rightarrow$Paste nodes and edges from clipboard}\\
\textbf{Plugins$\Rightarrow$BiNoM 2.0$\Rightarrow$BiNoM Utilities$\Rightarrow$Clipboard$\Rightarrow$Show clipboard contents}\\
\clearpage

\section{Appendices}
\subsection{Attributed graph model}\label{Attributed_graph_model}
BiNoM manipulates the information contained in the standard systems biology files by mapping it onto a labeled graph, called index. The index does not try to map the totality of all details; it rather serves as a connection map for the objects contained in other ontologies such as BioPAX. In other words, the index contains the minimum information needed to graphically represent objects and connections between them. Index elements (nodes and edges) are annotated by identifiers sufficient to find these objects in the original files and extract and edit the information related to them.\\\\
This approach has several advantages, in particular, with respect to synchronization issues. BiNoM index is a light-weight construction which can be easily regenerated, does not duplicate the information in existing files and serves only to facilitate the visualization and to access existing systems biology files.\\\\
Currently, BiNoM index is mostly developed to map BioPAX ontology files and CellDesigner object schema. In future versions, other mappings will be available, for instance, a mapping to SBML files annotated with Systems Biology Ontology (http://www.ebi.ac.uk/sbo/).\\\\
The table~\ref{Attribute_table} lists all attributes used by the index.
\begin{figure}
\centering
\includegraphics[width=0.8\textwidth]{graphics/Attribute_table}
\caption{All attributes of graph model used by the index}
\label{Attribute_table}
\end{figure}

\subsection{BiNoM CellDesigner and BiNoM BioPAX visual mappers}\label{CellDesigner_BioPAX_visual_mappers}
BiNoM has two built-in visual mappers supporting the visualization of the whole index or of its parts. The legend for deciphering the different types of visualization is provided in figure~\ref{BioPAX_visualizations}.
\begin{figure}
\centering
\includegraphics[width=0.8\textwidth]{graphics/BioPAX_visualizations}
\caption{Types of visualization in BioPAX and CellDesigner}
\label{BioPAX_visualizations}
\end{figure}

\subsection{BiNoM Naming Service}\label{BiNoM_Naming_Service}
When importing pathway information, BiNoM tries to generate meaningful, unique and short names for index entities. This function of the plugin is performed via BiNoM Naming Service. For proteins and other entities, the shortest available synonym is used. For genes, a ‘g’ symbol is added at the beginning of the name, and for RNAs, a ‘r’ symbol is added in order to avoid mixing genes and mRNAs with their products. If this leads to an ambiguity, it is resolved by adding a suffix specifying a unique id of the entity.\\\\
A chemical species in BiNoM is defined as a physical entity (such as protein) with some cellular localization and some (post-translational) modification (possibly none). The general template of the species label is the following:\\
Entity1\_name\textbar Modification1\textbar Modification2\textbar…: Entity2\_name\textbar Modifications...[\_active\textbar \_hmN]@compartment\\
Here, the colon symbol ‘:’ delimitates the different components of a complex if the species has several components. Optional suffixes ‘active’ or ‘hm’ describe active state of the chemical species or N-homodimer state, respectively.\\\\ Several examples of naming chemical species are presented:
\begin{itemize}
\item Naming chemical species shown in Systems Biology Graphical Notation standard figure~\ref{Names_in_SBGN_standard}
\item A conversion from CellDesigner figure~\ref{From_CellDesigner_to_Cytoscape}.
\item A conversion from BioPAX figure~\ref{Fragment_of_Apoptosis_from_Reactome}.
\end{itemize}
\begin{figure}
\centering
\includegraphics[width=0.8\textwidth]{graphics/Names_in_SBGN_standard}
\caption{2 examples of naming chemical species shown in Systems Biology Graphical Notation standard.}
\label{Names_in_SBGN_standard}
\end{figure}
\begin{figure}
\centering
\includegraphics[width=0.8\textwidth]{graphics/From_CellDesigner_to_Cytoscape}
\caption{Conversion of a little network from CellDesigner Graphical Notation to BiNoM index representation}
\label{From_CellDesigner_to_Cytoscape}
\end{figure}
\begin{figure}
\centering
\includegraphics[width=0.8\textwidth]{graphics/Fragment_of_Apoptosis_from_Reactome}
\caption{Small fragment of BioPAX index generated for Apoptosis pathway and extracted from Reactome database}
\label{Fragment_of_Apoptosis_from_Reactome}
\end{figure}

\subsection{Standard BioPAX interfaces}\label{Standard_BioPAX_Interfaces}
BiNoM index serves as a visual connector to the content of a network file. However, with all types of relations, the index is a highly connected graph and not very insightful when represented entirely. A subgraph of the index can be extracted according to a specific purpose and used to understand a specific aspect of the pathway information. We will call interface such a subgraph of the entire index.\\\\
When importing a BioPAX file, BiNoM proposes to generate three standard BioPAX interfaces referred to as
\nopagebreak
\begin{itemize}
\item Reaction Network.
\item Pathway Structure.
\item Protein-Protein Interaction.
\end{itemize}
\subsubsection{BioPAX interface as Reaction Network}
The Reaction Network interface is a bipartite graph which contains nodes of only two types: ‘species’ and ‘reactions’. Reactants are connected to reactions through edges of type LEFT, products are connected through edges of type RIGHT. Modifier species are connected through CATALYSIS, MODULATION and other edges. See figure~\ref{BioPAX_reaction_network}.\\\\
Some BioPAX objects (catalysis, for example) are represented by edges with the corresponding BIOPAX\_URI attribute.
A chemical species node can correspond to several grouped physicalEntityParticipants, thus, it can have several BIOPAX\_URI attributes. When calling BioPAX editor, all of them will be opened.\\\\
Standard Reaction Network interface can be exported to pure SBML format (level 2) and serve as a draft for further computational modeling.
\begin{figure}
\centering
\includegraphics[width=0.8\textwidth]{graphics/BioPAX_reaction_network}
\caption{Fragment of Apoptosis from Reactome as Reaction Network.}
\label{BioPAX_reaction_network}
\end{figure}
\subsubsection{BioPAX interface as Pathway Structure}
Pathway Structure interface contains only nodes of ‘pathway’, ‘pathwayStep’ and ‘interaction’ types. The types of the edges connecting them are ‘CONTAINS’, ‘STEP’ and ‘NEXT’.  See figure~\ref{BioPAX_pathway_structure}.
\begin{figure}
\centering
\includegraphics[width=0.8\textwidth]{graphics/BioPAX_pathway_structure}
\caption{Fragment of Apoptosis from Reactome as Pathway Structure.}
\label{BioPAX_pathway_structure}
\end{figure}
\subsubsection{BioPAX interface as Protein-Protein Interaction}
Protein-protein Interaction interface contains only entities (not chemical species) with edges of ‘CONTAINS’ and ‘physicalInteraction’ type. This interface allows to visualize the composition of complexes like the Caspase3 example of the Apoptosis pathway (left, first), or, explicit information about protein interaction with TGFB1 (left, second), as in the NetPath TGF-beta BioPAX file.  See figure~\ref{BioPAX_protein_protein_interaction}.
\begin{figure}
\centering
\includegraphics[width=0.8\textwidth]{graphics/BioPAX_protein_protein_interaction}
\caption{Fragment of Apoptosis from Reactome as Protein Protein Interaction.}
\label{BioPAX_protein_protein_interaction}
\end{figure}

\subsection{AIN file format} \label{AIN_file_format}
The AIN format describes a list of influences between genes, proteins, modified proteins or families. It is a table in ASCII, where the columns are separated by one tabulation (\textless Tab\textgreater) .\\\\
The first line must start with the name of each column as follows (the titles are fixed):\\
ReviewRef ExperimentRef Link ChemType Delay Confidence Tissue Comment\\
(each space corresponds to a \textless Tab\textgreater on your keyboard).
\begin{itemize}
\item For the references (ReviewRef and ExperimentRef), if one wants to include a PUBMED number, it should have the form “PMID:123456.
\item The Link column describes a connection (activation or inhibition) between two entities, like “A-\textgreater B” or “A-\textbar B”. The entities can be simply the name of a gene or a protein, but it can also be a complex (“(C:D)”), a phosphorylated protein (“(C\textasciicircum p)”) or a family. In the latter case, the family can be given explicitly by the list of all its members (“(C1,C2,C3)”) or implicitly, by un undefined name (“(C.)”), where the “.” can be replaced by any character..
\item In the other columns, if the user wishes to add more than one word in each field, the sentences need to be inserted between “…”.
\item If a field cannot be filled, a simple dot should be inserted.
\item A \# in first column makes the line comment.
\end{itemize}\
For an example of AIN format, one can open the file ExamplApop.txt in a simple text editor or in spreadsheet as EXCEL. All the information in this AIN file is translated in BioPAX format when the file is imported in Cytoscape via BiNoM.

\subsection{Modularization by shortest path clustering}\label{Agglomeration_by_shortest_path}
When only the structure of a network is known, the simplest method to agglomerate nodes in a network is to put the closest nodes together. And so modules may have the fewest links between them. This method can lead to an algorithm of modularization of an oriented network. The notion of closeness and the process of creating modules are to be clarified.\\\\
The distance between nodes is based on the length of the shortest paths and the number of occurrences if several paths are equal (the equality of the shortest path is frequent in a strongly connected network). The distance from node 1 to node 2 is generally different from distance from node 2 to node 1.\\\\
The used distance is the minimal linkage applied to the base distance, which is necessary to respect the triangular inequality. The distance between A and B is the minimum of distances from nodes in A to nodes in B and from nodes in B to nodes in A. And so, the agglomerative hierarchical clustering can be applied to build modules.\\\\
To avoid too speed increasing of clusters, they are ranked in a queue and the last created cluster is put at the end of the queue. For the same reason, nodes are sorted by in degree (sources in first). Despite of these precautions, the algorithm applied to strongly connected network gives unbalanced clusters (often a hudge cluster and several tiny clusters). So, a ceiling number of nodes in a cluster must be fixed.\\\\
The agglomerative clustering gives 1 cluster at the end, which has no interest. That’s why; these 2 stop conditions are added:
\begin{itemize}
\item The length of the shortest path between 2 clusters reaches the maximal length.
\item The number of clusters to be compared in the queue is less than 2.
\end{itemize}
The first stop condition make that too far clusters are not merged. When the last cluster to be created contains more than the maximal number of nodes, the largest cluster is excluded from the queue. Only the clusters remaining in the queue are to be compared by distance and they must be 2 or more.\\\\
The next page shows 3 examples (network inspired by toynet ). If the maximal length of the shortest paths is 1, nodes inside clusters are connected as a clique in a not oriented graph. But, if not, it may not be the case.\\\\
From a \textbf{practical point of view}, the input of ceiling number of nodes and maximal length of the shortest paths gives a set of not intersecting sub-networks. They are a partition of the network; their union is the whole network. This process is only useful for connected networks. Obviously, isolated nodes or sub-networks are not merged unless the maximal distance is infinity.

\subsection{GLOSSARY}\label{GLOSSARY}
\subsubsection{BioPAX}
BioPAX is an OWL (Web Ontology Language) document designed to exchange biological pathways. BioPAX format provides separate layers of information: information about the reactions involved in the networks along with the participants, information about the structure of the pathway, and information about the protein-protein interactions. 
\subsubsection{CellDesigner}
CellDesigner is a structured diagram editor for drawing gene-regulatory and biochemical networks. Networks are drawn based on the process diagram, with graphical notation system proposed by Kitano.
\subsubsection{BiNoM Index}
Directed labeled graph representing the objects in CellDesigner and BioPAX ontologies and their connections. Index maps only the information needed to display it and to identify the relevant information in the original CellDesigner or BioPAX files.
\subsubsection{BiNoM interface}
Part of the BiNoM index (subgraph) visually presented by Cytoscape network. There are standard interfaces (Reaction network, pathway structure, protein interaction) which can be combined to construct a user-defined interface.
\subsubsection{Optimal / suboptimal shortest paths}
Shortest paths in weighted directed graph – paths in the graph between source and target nodes with minimal sum of weights of the edges making the path. Suboptimal path is constructed by removing all edges in all shortest paths one by one and one at a time and finding the shortest path.
\subsubsection{Strongly Connected Components (SCC)}
A subgraph in a directed graph, in which there is path from any node to any node
\subsubsection{Relevant cycle}
Any cyclic path in the graph which can not be decomposed further into simpler cycles
\subsubsection{SBML}
Systems Biology Markup Language (SBML) is a standard for representing models of biochemical and gene-regulatory networks.


.

\clearpage
\bibliographystyle{plain}
\bibliography{BiNoM_Manual_v2}

\end{document}
